\chapter{Implementation}

\section{Honeypot Plugin Framework}

\subsection{Plugin Manager}

The honeypot plugin manager is implemneted using the Go programming language.
Communication with plugins is done via the Go standard library's RPC module.

\subsection{Honeypot Plugins}

Each honeypot plugin is also implemented in the Go programming language.

\subsubsection{SSH Plugin}

The SSH plugin uses Go's \texttt{crypto/ssh} package to simulate legitimate SSH
protocol handshakes with two custom callbacks that log alerts when an attacker
attempts either a password-based login or a key-based login.

\subsubsection{HTTP Plugin}

The HTTP plugin uses Go's \texttt{net/http} package to implement a standard HTTP
server, along with Go's \texttt{html/template} package to provide site- and
device-specific template configuration for the fake login page.

\subsubsection{HTTPS Plugin}

Along with the packages used for the HTTP plugin, the HTTPS plugin also uses
Go's \texttt{http.ListenAndServeTLS} module to handle TLS protocol negotiation
and operation.

\subsubsection{DNP3 Plugin}

The DNP3 plugin uses a custom implementation of DNP3 protocol parsing. In order
to accomodate the many different types of DNP3 protocol messages, a SecureAuth
challenge is sent in response to any incoming command message, and the challenge
is always rejected. This allows a full DNP3 monitoring implementation without
needing to implement any of the protocol state or controlling logic.

\section{Supporting System}

\subsection{Configuration Management}

The Ansible configuration management system is used to provide an automated
means of configuring, updating, and deploying many simultaneous device installations.
A set of configurable Ansible ``roles'' is included as part of the final deliverable
that allow our client to tailor the system to their specific needs automatically.

\subsection{Firewall}

The device firewall uses the standard linux \texttt{iptables} package.

\subsection{Administration SSH Server}

The core device administration server uses the standard \texttt{OpenSSH} server
implementation.

\subsection{Intrusion Detection System}

The intrusion detection system included on the device uses the industry-standard
Snort IDS, along with a minimal set of initial rules tailored to the client's
operating environment. Future changes and and ruleset adjustments for the IDS
can be made using the Ansible configuration mangement system described above.
