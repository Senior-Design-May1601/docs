\chapter{Design and Implementation}

\section{Honeypot Plugin Framework}

\subsection{Motivation}

Our decision to implement a plugin framework for honeypots instead of a
static, monolithic program tailored exactly to our client's current needs
comes from three key observations:

\begin{enumerate}
    \item Overall client needs evolve and change over time
    \item Each substation may require a different set of honeypot protocols
    \item Large programs implementing many different network protocols become
    very hard to reason about and test
\end{enumerate}

By using a modular, extensible architecture from the beginning, we are able
to react quickly to evolving client specifications as well as lay the groundwork
for future, proprietary or site-specific extensions to be made.

\subsection{Plugin Architecture}

Figure \ref{fig:hpot-arch} shows how honeypot plugins, logging modules, and
the core manager fit together in a single system.

\begin{center}
\begin{figure}
{\scalefont{0.9}
\scalebox{0.7}{% Graphic for TeX using PGF
% Title: /home/nskinkel/src/may1601/docs/spring2016/final-presentation/dia/hpot-arch.dia
% Creator: Dia v0.97.3
% CreationDate: Thu Apr  7 01:55:27 2016
% For: nskinkel
% \usepackage{tikz}
% The following commands are not supported in PSTricks at present
% We define them conditionally, so when they are implemented,
% this pgf file will use them.
\ifx\du\undefined
  \newlength{\du}
\fi
\setlength{\du}{15\unitlength}
\begin{tikzpicture}
\pgftransformxscale{1.000000}
\pgftransformyscale{-1.000000}
\definecolor{dialinecolor}{rgb}{0.000000, 0.000000, 0.000000}
\pgfsetstrokecolor{dialinecolor}
\definecolor{dialinecolor}{rgb}{1.000000, 1.000000, 1.000000}
\pgfsetfillcolor{dialinecolor}
\pgfsetlinewidth{0.100000\du}
\pgfsetdash{}{0pt}
\pgfsetdash{}{0pt}
\pgfsetmiterjoin
\definecolor{dialinecolor}{rgb}{1.000000, 1.000000, 1.000000}
\pgfsetfillcolor{dialinecolor}
\fill (5.000000\du,5.000000\du)--(5.000000\du,7.000000\du)--(9.000000\du,7.000000\du)--(9.000000\du,5.000000\du)--cycle;
\definecolor{dialinecolor}{rgb}{0.000000, 0.000000, 0.000000}
\pgfsetstrokecolor{dialinecolor}
\draw (5.000000\du,5.000000\du)--(5.000000\du,7.000000\du)--(9.000000\du,7.000000\du)--(9.000000\du,5.000000\du)--cycle;
\pgfsetlinewidth{0.100000\du}
\pgfsetdash{}{0pt}
\pgfsetdash{}{0pt}
\pgfsetmiterjoin
\definecolor{dialinecolor}{rgb}{1.000000, 1.000000, 1.000000}
\pgfsetfillcolor{dialinecolor}
\fill (5.000000\du,8.000000\du)--(5.000000\du,10.000000\du)--(9.000000\du,10.000000\du)--(9.000000\du,8.000000\du)--cycle;
\definecolor{dialinecolor}{rgb}{0.000000, 0.000000, 0.000000}
\pgfsetstrokecolor{dialinecolor}
\draw (5.000000\du,8.000000\du)--(5.000000\du,10.000000\du)--(9.000000\du,10.000000\du)--(9.000000\du,8.000000\du)--cycle;
\pgfsetlinewidth{0.100000\du}
\pgfsetdash{}{0pt}
\pgfsetdash{}{0pt}
\pgfsetmiterjoin
\definecolor{dialinecolor}{rgb}{1.000000, 1.000000, 1.000000}
\pgfsetfillcolor{dialinecolor}
\fill (5.000000\du,11.000000\du)--(5.000000\du,13.000000\du)--(9.000000\du,13.000000\du)--(9.000000\du,11.000000\du)--cycle;
\definecolor{dialinecolor}{rgb}{0.000000, 0.000000, 0.000000}
\pgfsetstrokecolor{dialinecolor}
\draw (5.000000\du,11.000000\du)--(5.000000\du,13.000000\du)--(9.000000\du,13.000000\du)--(9.000000\du,11.000000\du)--cycle;
\pgfsetlinewidth{0.100000\du}
\pgfsetdash{}{0pt}
\pgfsetdash{}{0pt}
\pgfsetmiterjoin
\definecolor{dialinecolor}{rgb}{1.000000, 1.000000, 1.000000}
\pgfsetfillcolor{dialinecolor}
\fill (5.000000\du,19.000000\du)--(5.000000\du,21.000000\du)--(9.000000\du,21.000000\du)--(9.000000\du,19.000000\du)--cycle;
\definecolor{dialinecolor}{rgb}{0.000000, 0.000000, 0.000000}
\pgfsetstrokecolor{dialinecolor}
\draw (5.000000\du,19.000000\du)--(5.000000\du,21.000000\du)--(9.000000\du,21.000000\du)--(9.000000\du,19.000000\du)--cycle;
\pgfsetlinewidth{0.100000\du}
\pgfsetdash{}{0pt}
\pgfsetdash{}{0pt}
\pgfsetmiterjoin
\definecolor{dialinecolor}{rgb}{1.000000, 1.000000, 1.000000}
\pgfsetfillcolor{dialinecolor}
\fill (15.000000\du,9.000000\du)--(15.000000\du,17.000000\du)--(22.000000\du,17.000000\du)--(22.000000\du,9.000000\du)--cycle;
\definecolor{dialinecolor}{rgb}{0.000000, 0.000000, 0.000000}
\pgfsetstrokecolor{dialinecolor}
\draw (15.000000\du,9.000000\du)--(15.000000\du,17.000000\du)--(22.000000\du,17.000000\du)--(22.000000\du,9.000000\du)--cycle;
\pgfsetlinewidth{0.100000\du}
\pgfsetdash{}{0pt}
\pgfsetdash{}{0pt}
\pgfsetmiterjoin
\definecolor{dialinecolor}{rgb}{1.000000, 1.000000, 1.000000}
\pgfsetfillcolor{dialinecolor}
\fill (28.000000\du,5.000000\du)--(28.000000\du,7.000000\du)--(32.000000\du,7.000000\du)--(32.000000\du,5.000000\du)--cycle;
\definecolor{dialinecolor}{rgb}{0.000000, 0.000000, 0.000000}
\pgfsetstrokecolor{dialinecolor}
\draw (28.000000\du,5.000000\du)--(28.000000\du,7.000000\du)--(32.000000\du,7.000000\du)--(32.000000\du,5.000000\du)--cycle;
\pgfsetlinewidth{0.100000\du}
\pgfsetdash{}{0pt}
\pgfsetdash{}{0pt}
\pgfsetmiterjoin
\definecolor{dialinecolor}{rgb}{1.000000, 1.000000, 1.000000}
\pgfsetfillcolor{dialinecolor}
\fill (28.000000\du,8.000000\du)--(28.000000\du,10.000000\du)--(32.000000\du,10.000000\du)--(32.000000\du,8.000000\du)--cycle;
\definecolor{dialinecolor}{rgb}{0.000000, 0.000000, 0.000000}
\pgfsetstrokecolor{dialinecolor}
\draw (28.000000\du,8.000000\du)--(28.000000\du,10.000000\du)--(32.000000\du,10.000000\du)--(32.000000\du,8.000000\du)--cycle;
\pgfsetlinewidth{0.100000\du}
\pgfsetdash{}{0pt}
\pgfsetdash{}{0pt}
\pgfsetmiterjoin
\definecolor{dialinecolor}{rgb}{1.000000, 1.000000, 1.000000}
\pgfsetfillcolor{dialinecolor}
\fill (28.000000\du,11.000000\du)--(28.000000\du,13.000000\du)--(32.000000\du,13.000000\du)--(32.000000\du,11.000000\du)--cycle;
\definecolor{dialinecolor}{rgb}{0.000000, 0.000000, 0.000000}
\pgfsetstrokecolor{dialinecolor}
\draw (28.000000\du,11.000000\du)--(28.000000\du,13.000000\du)--(32.000000\du,13.000000\du)--(32.000000\du,11.000000\du)--cycle;
\pgfsetlinewidth{0.100000\du}
\pgfsetdash{}{0pt}
\pgfsetdash{}{0pt}
\pgfsetmiterjoin
\definecolor{dialinecolor}{rgb}{1.000000, 1.000000, 1.000000}
\pgfsetfillcolor{dialinecolor}
\fill (28.000000\du,19.000000\du)--(28.000000\du,21.000000\du)--(32.000000\du,21.000000\du)--(32.000000\du,19.000000\du)--cycle;
\definecolor{dialinecolor}{rgb}{0.000000, 0.000000, 0.000000}
\pgfsetstrokecolor{dialinecolor}
\draw (28.000000\du,19.000000\du)--(28.000000\du,21.000000\du)--(32.000000\du,21.000000\du)--(32.000000\du,19.000000\du)--cycle;
\pgfsetlinewidth{0.100000\du}
\pgfsetdash{}{0pt}
\pgfsetdash{}{0pt}
\pgfsetbuttcap
{
\definecolor{dialinecolor}{rgb}{0.000000, 0.000000, 0.000000}
\pgfsetfillcolor{dialinecolor}
% was here!!!
\pgfsetarrowsend{latex}
\definecolor{dialinecolor}{rgb}{0.000000, 0.000000, 0.000000}
\pgfsetstrokecolor{dialinecolor}
\draw (9.000000\du,6.000000\du)--(14.950256\du,10.384399\du);
}
\pgfsetlinewidth{0.100000\du}
\pgfsetdash{}{0pt}
\pgfsetdash{}{0pt}
\pgfsetbuttcap
{
\definecolor{dialinecolor}{rgb}{0.000000, 0.000000, 0.000000}
\pgfsetfillcolor{dialinecolor}
% was here!!!
\pgfsetarrowsend{latex}
\definecolor{dialinecolor}{rgb}{0.000000, 0.000000, 0.000000}
\pgfsetstrokecolor{dialinecolor}
\draw (9.000000\du,9.000000\du)--(14.950256\du,11.505371\du);
}
\pgfsetlinewidth{0.100000\du}
\pgfsetdash{}{0pt}
\pgfsetdash{}{0pt}
\pgfsetbuttcap
{
\definecolor{dialinecolor}{rgb}{0.000000, 0.000000, 0.000000}
\pgfsetfillcolor{dialinecolor}
% was here!!!
\pgfsetarrowsend{latex}
\definecolor{dialinecolor}{rgb}{0.000000, 0.000000, 0.000000}
\pgfsetstrokecolor{dialinecolor}
\draw (9.000000\du,12.000000\du)--(15.000000\du,13.000000\du);
}
\pgfsetlinewidth{0.100000\du}
\pgfsetdash{}{0pt}
\pgfsetdash{}{0pt}
\pgfsetbuttcap
{
\definecolor{dialinecolor}{rgb}{0.000000, 0.000000, 0.000000}
\pgfsetfillcolor{dialinecolor}
% was here!!!
\pgfsetarrowsend{latex}
\definecolor{dialinecolor}{rgb}{0.000000, 0.000000, 0.000000}
\pgfsetstrokecolor{dialinecolor}
\draw (9.000000\du,20.000000\du)--(14.950256\du,14.615601\du);
}
\pgfsetlinewidth{0.100000\du}
\pgfsetdash{}{0pt}
\pgfsetdash{}{0pt}
\pgfsetbuttcap
{
\definecolor{dialinecolor}{rgb}{0.000000, 0.000000, 0.000000}
\pgfsetfillcolor{dialinecolor}
% was here!!!
\pgfsetarrowsend{latex}
\definecolor{dialinecolor}{rgb}{0.000000, 0.000000, 0.000000}
\pgfsetstrokecolor{dialinecolor}
\draw (22.049744\du,10.384399\du)--(28.000000\du,6.000000\du);
}
\pgfsetlinewidth{0.100000\du}
\pgfsetdash{}{0pt}
\pgfsetdash{}{0pt}
\pgfsetbuttcap
{
\definecolor{dialinecolor}{rgb}{0.000000, 0.000000, 0.000000}
\pgfsetfillcolor{dialinecolor}
% was here!!!
\pgfsetarrowsend{latex}
\definecolor{dialinecolor}{rgb}{0.000000, 0.000000, 0.000000}
\pgfsetstrokecolor{dialinecolor}
\draw (22.049744\du,11.505371\du)--(28.000000\du,9.000000\du);
}
\pgfsetlinewidth{0.100000\du}
\pgfsetdash{}{0pt}
\pgfsetdash{}{0pt}
\pgfsetbuttcap
{
\definecolor{dialinecolor}{rgb}{0.000000, 0.000000, 0.000000}
\pgfsetfillcolor{dialinecolor}
% was here!!!
\pgfsetarrowsend{latex}
\definecolor{dialinecolor}{rgb}{0.000000, 0.000000, 0.000000}
\pgfsetstrokecolor{dialinecolor}
\draw (22.049744\du,12.626343\du)--(28.000000\du,12.000000\du);
}
\pgfsetlinewidth{0.100000\du}
\pgfsetdash{}{0pt}
\pgfsetdash{}{0pt}
\pgfsetbuttcap
{
\definecolor{dialinecolor}{rgb}{0.000000, 0.000000, 0.000000}
\pgfsetfillcolor{dialinecolor}
% was here!!!
\pgfsetarrowsend{latex}
\definecolor{dialinecolor}{rgb}{0.000000, 0.000000, 0.000000}
\pgfsetstrokecolor{dialinecolor}
\draw (22.049744\du,15.615601\du)--(28.000000\du,20.000000\du);
}
% setfont left to latex
\definecolor{dialinecolor}{rgb}{0.000000, 0.000000, 0.000000}
\pgfsetstrokecolor{dialinecolor}
\node[anchor=west] at (5.900000\du,6.050000\du){SSH};
% setfont left to latex
\definecolor{dialinecolor}{rgb}{0.000000, 0.000000, 0.000000}
\pgfsetstrokecolor{dialinecolor}
\node[anchor=west] at (5.800000\du,9.050000\du){HTTP};
% setfont left to latex
\definecolor{dialinecolor}{rgb}{0.000000, 0.000000, 0.000000}
\pgfsetstrokecolor{dialinecolor}
\node[anchor=west] at (5.600000\du,12.000000\du){HTTPS};
% setfont left to latex
\definecolor{dialinecolor}{rgb}{0.000000, 0.000000, 0.000000}
\pgfsetstrokecolor{dialinecolor}
\node[anchor=west] at (5.600000\du,20.000000\du){DNP3};
% setfont left to latex
\definecolor{dialinecolor}{rgb}{0.000000, 0.000000, 0.000000}
\pgfsetstrokecolor{dialinecolor}
\node[anchor=west] at (28.650000\du,6.050000\du){Splunk};
% setfont left to latex
\definecolor{dialinecolor}{rgb}{0.000000, 0.000000, 0.000000}
\pgfsetstrokecolor{dialinecolor}
\node[anchor=west] at (28.750000\du,9.100000\du){Syslog};
% setfont left to latex
\definecolor{dialinecolor}{rgb}{0.000000, 0.000000, 0.000000}
\pgfsetstrokecolor{dialinecolor}
\node[anchor=west] at (28.350000\du,12.050000\du){Text File};
% setfont left to latex
\definecolor{dialinecolor}{rgb}{0.000000, 0.000000, 0.000000}
\pgfsetstrokecolor{dialinecolor}
\node[anchor=west] at (28.950000\du,20.000000\du){S3};
% setfont left to latex
\definecolor{dialinecolor}{rgb}{0.000000, 0.000000, 0.000000}
\pgfsetstrokecolor{dialinecolor}
\node[anchor=west] at (16.600000\du,12.750000\du){Controller};
\pgfsetlinewidth{0.100000\du}
\pgfsetdash{{1.000000\du}{1.000000\du}}{0\du}
\pgfsetdash{{1.000000\du}{1.000000\du}}{0\du}
\pgfsetmiterjoin
\definecolor{dialinecolor}{rgb}{1.000000, 1.000000, 1.000000}
\pgfsetfillcolor{dialinecolor}
\fill (-1.000000\du,10.000000\du)--(-1.000000\du,12.000000\du)--(3.000000\du,12.000000\du)--(3.000000\du,10.000000\du)--cycle;
\definecolor{dialinecolor}{rgb}{0.000000, 0.000000, 0.000000}
\pgfsetstrokecolor{dialinecolor}
\draw (-1.000000\du,10.000000\du)--(-1.000000\du,12.000000\du)--(3.000000\du,12.000000\du)--(3.000000\du,10.000000\du)--cycle;
\pgfsetlinewidth{0.100000\du}
\pgfsetdash{{1.000000\du}{1.000000\du}}{0\du}
\pgfsetdash{{1.000000\du}{1.000000\du}}{0\du}
\pgfsetmiterjoin
\definecolor{dialinecolor}{rgb}{1.000000, 1.000000, 1.000000}
\pgfsetfillcolor{dialinecolor}
\fill (34.000000\du,10.000000\du)--(34.000000\du,12.000000\du)--(38.000000\du,12.000000\du)--(38.000000\du,10.000000\du)--cycle;
\definecolor{dialinecolor}{rgb}{0.000000, 0.000000, 0.000000}
\pgfsetstrokecolor{dialinecolor}
\draw (34.000000\du,10.000000\du)--(34.000000\du,12.000000\du)--(38.000000\du,12.000000\du)--(38.000000\du,10.000000\du)--cycle;
\pgfsetlinewidth{0.100000\du}
\pgfsetdash{{1.000000\du}{1.000000\du}}{0\du}
\pgfsetdash{{1.000000\du}{1.000000\du}}{0\du}
\pgfsetbuttcap
{
\definecolor{dialinecolor}{rgb}{0.000000, 0.000000, 0.000000}
\pgfsetfillcolor{dialinecolor}
% was here!!!
\pgfsetarrowsend{latex}
\definecolor{dialinecolor}{rgb}{0.000000, 0.000000, 0.000000}
\pgfsetstrokecolor{dialinecolor}
\pgfpathmoveto{\pgfpoint{1.124590\du}{12.049343\du}}
\pgfpatharc{168}{81}{5.150055\du and 5.150055\du}
\pgfusepath{stroke}
}
\pgfsetlinewidth{0.100000\du}
\pgfsetdash{{1.000000\du}{1.000000\du}}{0\du}
\pgfsetdash{{1.000000\du}{1.000000\du}}{0\du}
\pgfsetbuttcap
{
\definecolor{dialinecolor}{rgb}{0.000000, 0.000000, 0.000000}
\pgfsetfillcolor{dialinecolor}
% was here!!!
\pgfsetarrowsend{latex}
\definecolor{dialinecolor}{rgb}{0.000000, 0.000000, 0.000000}
\pgfsetstrokecolor{dialinecolor}
\pgfpathmoveto{\pgfpoint{29.999585\du}{15.999954\du}}
\pgfpatharc{97}{16}{5.413252\du and 5.413252\du}
\pgfusepath{stroke}
}
% setfont left to latex
\definecolor{dialinecolor}{rgb}{0.000000, 0.000000, 0.000000}
\pgfsetstrokecolor{dialinecolor}
\node[anchor=west] at (-0.850000\du,10.700000\du){Custom};
% setfont left to latex
\definecolor{dialinecolor}{rgb}{0.000000, 0.000000, 0.000000}
\pgfsetstrokecolor{dialinecolor}
\node[anchor=west] at (-0.850000\du,11.500000\du){Honeypot};
% setfont left to latex
\definecolor{dialinecolor}{rgb}{0.000000, 0.000000, 0.000000}
\pgfsetstrokecolor{dialinecolor}
\node[anchor=west] at (34.200000\du,10.700000\du){Custom};
% setfont left to latex
\definecolor{dialinecolor}{rgb}{0.000000, 0.000000, 0.000000}
\pgfsetstrokecolor{dialinecolor}
\node[anchor=west] at (34.200000\du,11.500000\du){Logger};
% setfont left to latex
\definecolor{dialinecolor}{rgb}{0.000000, 0.000000, 0.000000}
\pgfsetstrokecolor{dialinecolor}
\node[anchor=west] at (10.600000\du,6.900000\du){Application-specific alerts broadcasted to loggers};
\end{tikzpicture}
}
}
\label{fig:hpot-arch}
\caption{Honepot Plugin Architecture}
\end{figure}
\end{center}

\subsubsection{What is a Plugin?}

There are two types of recognized plugins: Honeypot Plugins and Logging Plugins.

Honeypot plugins are small, isolated programs that simply import our logging library
(which uses exactly the same interface as the standard Go logging library)
and log any events that should be broadcast as alerts (for instance, an
attempted SSH login).

Logging plugins must implement a single-function interface that specifies the
behavior that should take place when the logger receives an alert (for instance,
writing the byte string to a text file).

Each plugin is responsible for implementing its own, protocol-specific functionality
(for instance, an HTTPS honeypot plugin needs to do TLS protocol negotiation and handshakes
as well as respond to HTTP requests on its own, and a Splunk logging plugin needs
to coordinate with the remote endpoint independently), and the only interface to
the rest of the system is through the imported logging module for honeypot
plugins and the single-function interface for logging plugins.

\subsubsection{Plugin Management}

The plugin manager runs as a system daemon and is responsible for starting,
stopping, and managing all plugins as child subprocesses.
The plugin manager receives alerts from honeypot plugins and broadcasts them to
each logging plugin. The plugin manager also centralizes error messages and
reporting and provides a single interface to configure the system.

\subsubsection{Plugin Communication}

Plugins communicate via an asynchronous message-passing protocol which uses
Go's RPC library under the hood. When honeypot plugins call an imported logging
function, an asynchronous RPC call is made behind the scenes to the plugin
manager, using a connection that is setup and initiated when the logger is
created.

Logger plugins also receive messages over via the same message-passing protocol,
and the interface they implement is actually the method called via asynchronous
RPC.

\subsubsection{The Default Plugin Set}

Along with the plugin architecture itself, our deliverable includes a set
of default plugins tailored to our client's current needs. This set includes
the honeypot plugins for the following protocols:
\begin{itemize}
    \item \textbf{SSH}: the SSH plugin simulates and SSH protocol information and
    logs metadata about the attempted login
    \item \textbf{HTTP}: the HTTP plugin presents a fake login page and harvests
    attacker credential
    \item \textbf{HTTPS}: similar in practice to the HTTP plugin, the HTTPS plugin
    adds TLS to the fake login page
    \item \textbf{DNP3}: the DNP3 plugin collects DNP3 SCADA commands and logs
    attempted attacker activity
\end{itemize}

The default plugin set also includes the following logging plugins:
\begin{itemize}
    \item \textbf{Splunk}: the Splunk logger logs alerts to a remote Splunk
    instance using the Splunk Event Collector API
    \item \textbf{TextFile}: the text file logger logs alerts to a local text
    file
    \item \textbf{Syslog}: the Syslog logger logs syslog alerts on the device
\end{itemize}


\section{Supporting System Design}

The supporting system is essentially every device component that is not part of
the honeypot plugin framework. For instance: the operating system itself,
particular critical software components (e.g. iptables), and the intrusion
detection system.

\subsection{Design Considerations}

The supporting system is designed to accomodate our client's particular needs
as well as minimize the amount of ongoing maintenance and system-specific work
that must be performed.

We use the Raspbian operating system, designed for Raspberry Pi's, as the base
system. All other components are designed to stay as close to the upstream
packages as possible (minimize or eliminate custom modifications).

\subsection{System Components}

The supporting system has the following critical components, described in more
detail in the Implementation Section:
\begin{enumerate}
    \item Firewall
    \item (Real) SSH server for ongoing maintenance
    \item Intrusion Detection System
\end{enumerate}
