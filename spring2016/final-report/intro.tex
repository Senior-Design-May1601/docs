\chapter{Project Overview}

\section{Purpose}

The purpose of this document is to act as a final deliverable for the ICS/SCADA Traffic Baseline and Honeypot project.  A brief description of the project will be provided, followed by a detailed layout of the final product design, measures taken to ensure adequate testing, and guidelines concerning proper implementation of the device. Appendices will be used to deliver a step-by-step guide to device deployment and maintenance, as well as project concerns and considerations that should be noted.

\section{Background}

Supervisory Control and Data Acquisition (SCADA) systems play a vital role in
maintaining and controlling critical infrastructure such as
water treatment plants, oil pipelines, HVAC Systems, and the power grid.
The societal importance of these services makes them a high-value target
for well-funded, highly-skilled adversaries who increasingly turn to digital
and electronic attack vectors, exploiting old and outdated protocols that
were never designed to be exposed to the public internet.
Thus, rapid intrusion detection and response is crucial for safeguarding
SCADA networks.

A \textit{honeypot} (a system designed to mimic a legitimate protocol, e.g.
SSH, coerce an attacker into interacting with it, and then report attacker
activity  to an administrator) is an old and well-understood approach to
detecting network intruders.

\subsection{Problem}

Our client, Alliant Energy, requested many small, low-maintenance honeypot
systems,
each capable of speaking multiple network protocols (both SCADA and non-SCADA),
able to log to many different backends, and able to sniff local network traffic
for anomalies, to place inside each of their 28 electrical substations.

Unfortunately, many mainstream honeypot implementations have a number of
problems that make them unsuitable for deployment in Alliant's substations,
including:

\begin{itemize}
\setlength{\columnsep}{0cm}
\def\columnseprulecolor{}
\item\textbf{Unsafe Languages:} Many existing honeypots rely on code written in languages without memory- or type-safety, unacceptable for a high-security environment.

\item\textbf{Single-Protocol:} Most honeypots are designed to speak only a single protocol.

\item\textbf{Complex Deployment:} Open source honepots are generally not designed to be deployed, updated, and configured en masse.

\item\textbf{Expensive Hardware:} Commercial honeypots are too expensive to place in multiple locations across many subnets.
\end{itemize}

\subsection{Solution}

We designed a honeypot \textit{plugin framework}
to enforce high-security process isolation, extreme extensibility, and
easy and comprehensive testing. The system is tailored to run on Raspberry
Pi consumer hardware and is cheap enough to buy in quantity, while
still being powerful enough to run an Intrusion Detection System (IDS) capable
of watching local network traffic for anomalies. Finally, we delivered a set
of configuration management utilities that allow our client to deploy to
many different locations and update all devices in a \textit{single-step}.

\section{Deliverables}

The final deliverable of this project will be based upon the assumption that the client will be required to provide the following hardware per honeypot device deployed.

\subsection{Hardware}
\begin{itemize}
\item RaspberryPi Model 2 B
\item 8GB MicroSD card
\item 2.5A Micro USB power supply(5ft cable) with noise filter.
\item USB 3.0 network adapter to RJ45 Ethernet connection.
\item RaspberryPi plastic hardcase.
\item Appropriate ethernet connection
\end{itemize}

\subsection{Software}
All code necessary to deploy our device can be cloned from the git repository address: \url{https://github.com/Senior-Design-May1601/config}.

An iso image of the full system deployed on a Raspbian operating system will also be provided, however it is highly recommended that the client deploy devices via Ansible as to streamline the process and apply proper configurations per device.
