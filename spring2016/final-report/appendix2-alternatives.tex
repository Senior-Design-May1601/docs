\chapter{Alternatives}

\section{Hardware}
When coming up with the initial design for Alliant's low interaction honeypot there were two viable options. The first option was to build a device using standard computer hardware. The second was to use a prefabricated single-board computer such as a Raspberry Pi. Both options are viable for creating a functional honeypot with pros and cons associated with each. A custom built device would contain more processing power, more RAM and have a high degree of customization. The drawback with a custom design is that they are more expensive and not as easily replaceable as a prefabricated device. This was the main draw towards a Raspberry Pi. These devices are relatively cheap, and offer the easy setup and installation required for deploying multiple remote machines. Ultimately the decision came down to what was the simplest solution that would allow for the use of SSH, HTTP, HTTPS, DNP3 an intrusion detection system and a means of logging. Since all of the services running  on the device are minimal versions it is ultimately unnecessary to create a custom machine with extended processing power and RAM. A single board computer such as a Raspberry Pi is more than capable of accomplishing the required tasks for less than half the cost of a custom built device while also requiring less manpower to setup and maintain. For these reasons A Raspberry Pi was chosen as the platform of choice for Alliant's SCADA honeypot system.
