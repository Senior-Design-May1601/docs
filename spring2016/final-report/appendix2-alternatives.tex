\chapter{Alternatives}

\section{Hardware}
When coming up with the initial design for Alliant's low interaction honeypot there were two viable options. The first option was to build a device using standard computer hardware. The second was to use a prefabricated single-board computer such as a Raspberry Pi. Both options are viable for creating a functional honeypot with pros and cons associated with each. A custom built device would contain more processing power, more RAM and have a high degree of customization. The drawback with a custom design is that they are more expensive and not as easily replaceable as a prefabricated device. This was the main draw towards a Raspberry Pi. These devices are relatively cheap, and offer the easy setup and installation required for deploying multiple remote machines. Ultimately the decision came down to what was the simplest solution that would allow for the use of SSH, HTTP, HTTPS, DNP3 an intrusion detection system and a means of logging. Since all of the services running  on the device are minimal versions it is ultimately unnecessary to create a custom machine with extended processing power and RAM. A single board computer such as a Raspberry Pi is more than capable of accomplishing the required tasks for less than half the cost of a custom built device while also requiring less manpower to setup and maintain. For these reasons A Raspberry Pi was chosen as the platform of choice for Alliant's SCADA honeypot system.

\section{Architecture}
During the first iterations of this project, the system architecture was singled-tiered. Initially, we intended to create one monolithic application. As we started to get further into development, we realized that many parts of this project were actually rather orthogonal to one another. At this point, a design idea was explored that pointed the project in the direction of our current plugin framework. This plugin framework became incredibly pragmatic as we continued development. The first reason this framework is ideal is because it greatly increases the security of the system. With the plugins running as separate processes, they are completely isolated from one another. This isolation also means they will be in completely different address space, which is a major benefit should there be any vulnerability or bug in any individual plugin. The other reason this framework is very fitting for this project is that it allows us to be highly extensible. With these plugins, should our client want to add some function, all they need to do is implement the main interface and they are immediately able to add plugins for new protocols or logging backends. The progression from our initial design to this plugin architecture has become a huge benefit not only to us during development and testing, but it will also be decidedly valuable for our client going forward.
