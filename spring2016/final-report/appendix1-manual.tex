\chapter{Operational Manual}

\section{Hardware Assembly}

After purchasing the hardware outlined in {ref deliverables hardware}, the assembly of the Raspberry Pi follows the standard vanilla installation instructions outlined by the Raspberry Pi Foundation at:
\\\newline
\textit{https://www.raspberrypi.org/documentation/installation/installing-images/README.md}
\\\newline
The device should be connected to the local ethernet twice, once using the Pi's on board ethernet connection, and a second time using the USB network interface adapter for the IDS connection. Remember to catalog the host name(IP Address) of each device for the next phase of this manual. Also note that these devices will need network connectivity in order to download software/packages during the deployment phase or upon administering updates.

\section{Deployment}
\subsection{Install Requirements}
The machine which will deploy/configure/maintain these honeypot devices will be required to have either an OS X or Linux operating system with the following packages installed. These packages can be installed using package managers like pipor apt which are often installed by default on these operating systems. 

The system should have the following minnimum software version.
\begin{itemize}
\item git(1.9.1)
\item ansible(2.0.2)
\item sshpass(1.05)
\end{itemize}

\subsection{Clone}
Once all pi devices are connected to there subsequent network environments and given local IP addresses, the installer can cd to an appropriate work directory and then use git to clone the repository at (TODO REF DELIVERABLE ADDRESS) 
\\\newline
\textit{\$ cd /home/user/ICS-Honeypot}
\newline
\textit{\$ git clone https://github.com/Senior-Design-May1601/config}
\\\newline

\subsection{Initialization}
After cloning the repository cd into the ansible directory
\\\newline
\textit{\$ cd config/ansible}
\\\newline
Now make the init shell script executable
\\\newline
\textit{\$ chmod +x init}
\\\newline
This shell script will do a number of things. First it will create 2 new directories
\begin{itemize}
\item \textbf{hosts:} stores ansible inventory files
\item \textbf{keys:} stores deployment SSH keys
\end{itemize}
Next the script will create a series of variables that will define the honeypot devices host name/login information. the default deployment user name is deploy.Passwords for root and deploy will be generated using (NIK PLEASE CLARIFY THIS).The script will prompt the user for an administrative IP address. This address should be the one of the current system which will deploy and maintain honeypot devices. After deployment, no other systems aside from the current host machine will be able to connect to honeypot devices via SSH.
\\The script will also prompt the user for an administrative email address which if entered will be used by the Apticron package on each machine to notify administration upon pertinent operating system security updates for the device. This email will not be used for general logging.
\\A Splunk HTTP-EVENT-COLLECTOR token value will need to be provided next, which will enable the device to log to an administrative Splunk server. Finally the script will encrypt all the data entered abaove into a secrets.yml variable file via ansible-vault which will be stored in \textbf{vars/secrets.yml}. The user should provide a password for this file which will be needed to run ansible playbooks in the future. Viewing the contents of this file or modifying it can be done by using the ansible-vault command as needed.

\subsection{System Configuration}

%TODO /hosts/hosts file configuration (IP Addresses)
%TODO IPtables rule-sets
%TODO config files located in /config/ansible/config/toml directory with appropriate names
%TODO IDS Ruleset

\subsection{Deploy}
With everything configured, the last step in deploying any number of honeypot devices is to simply run command
\\\newline
\textit{\$ ansible-playbook -i hosts/hosts bootstrap-playbook.yml}
\\\newline
This command will simultaneously bootstrap and start any number of devices at various locations by downloading necessary packages, applying system configurations, starting firewalls/IDS systems, and starting the core honeypot service. Be warned this will take some time as there are a small number of packages that need to be downloaded. Once the playbook has complete all devices will be setup as honeypot devices and will no longer require full network access until updates are required.

\section{Update}


