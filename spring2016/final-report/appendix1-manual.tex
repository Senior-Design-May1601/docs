\chapter{Operational Manual}

\section{Hardware Assembly}

After purchasing the hardware, the assembly of the Raspberry Pi follows the standard vanilla installation instructions outlined by the Raspberry Pi Foundation at: \url{https://www.raspberrypi.org/documentation/installation/installing-images/README.md}.


The device should be connected to the local ethernet twice, once using the Pi's on board ethernet connection, and a second time using the USB network interface adapter for the IDS connection. Remember to catalog the host name(IP Address) of each device for the next phase of this manual. Also note that these devices will need network connectivity in order to download software/packages during the deployment phase or upon administering updates.

\section{Deployment}
\subsection{Install Requirements}
The machine which will deploy/configure/maintain these honeypot devices will be required to have either an OS X or Linux operating system with the following packages installed. These packages can be installed using package managers like pipo r apt which are often installed by default on these operating systems. 

The system should have the following minnimum software version.
\begin{itemize}
\item git(1.9.1)
\item ansible(2.0.2)
\item sshpass(1.05)
\end{itemize}

\subsection{Clone}
Once all pi devices are connected to there subsequent network environments and given local IP addresses, the installer can cd to an appropriate work directory and then use git to clone the repository: \url{https://github.com/Senior-Design-May1601/config}.

\begin{figure}[H]
\small
\begin{lstlisting}[language=Bash,frame=single]
$ cd /home/user/ICS-Honeypot
$ git clone https://github.com/Senior-Design-May1601/config
\end{lstlisting}
\end{figure}

\subsection{Initialization}
After cloning the repository,\texttt{cd} to the
new directory and update permissions on the \texttt{init} file:

\begin{figure}[H]
\small
\begin{lstlisting}[language=Bash,frame=single]
$ cd config/ansible
$ chmod +x init
\end{lstlisting}
\end{figure}

This shell script will do a number of things. First it will create 2 new directories
\begin{itemize}
\item \textbf{hosts:} stores ansible inventory files
\item \textbf{keys:} stores deployment SSH keys
\end{itemize}
Next the script will create a series of variables that will define the honeypot devices host name/login information. The default deployment user name is deploy. Passwords for root and deploy will be generated using random characters. The script will prompt the user for an administrative IP address. This address should be the one of the current system which will deploy and maintain honeypot devices. After deployment, no other systems aside from the current host machine will be able to connect to honeypot devices via SSH.

The script will also prompt the user for an administrative email address which if entered will be used by the Apticron package on each machine to notify administration upon pertinent operating system security updates for the device. This email will not be used for general logging.

A Splunk HTTP-EVENT-COLLECTOR token value will need to be provided next, which will enable the device to log to an administrative Splunk server. Finally the script will encrypt all the data entered above into a secrets.yml variable file via ansible-vault which will be stored in \textbf{vars/secrets.yml}. The user should provide a password for this file which will be needed to run ansible playbooks in the future. Viewing the contents of this file or modifying it can be done by using the ansible-vault command as needed.

\subsection{Deploy}
With everything configured, the last step in deploying any number of honeypot devices is to simply run command

\begin{figure}[H]
\small
\begin{lstlisting}[language=Bash,frame=single]
$ ansible-playbook -i hosts/hosts bootstrap-playbook.yml
\end{lstlisting}
\end{figure}

This command will simultaneously bootstrap and start any number of devices at various locations by downloading necessary packages, applying system configurations, starting firewalls/IDS systems, and starting the core honeypot service. Be warned this will take some time as there are a small number of packages that need to be downloaded. Once the playbook has completed all devices will be setup as honeypot devices and will no longer require full network access until updates are required.

\section{Plugin Configuration}

Each honeypot has a simple configuration file it uses to control behavior and
plugin-specific settings. A sample configuration file, with a commented
explanation for each option, is included in this appendix. These sample
configuration files, to be used as examples, are also included in the
software deliverable.

\subsection{Plugin Manager}

\begin{figure}[H]
\begin{lstlisting}[language=Python,frame=single]
[MasterConfig]
# full path to the local error and event log
Logfile = "/path/to/log/file"

# each honeypot plugin gets an entry like this
[[PluginConfig]]
# the name of the plugin that should appear in log messages
Name = "plugin"
# full path to the plugin executable
Path = "/path/to/plugin"
# arguments the plugin should be started with
Args = ["-config", "/path/to/config-file"]

# each logging plugin gets an entry like this
[[LoggerConfig]]
# the name of the logger that should appear in local log messages
Name = "logger"
# full path to the logger executable
Path = "/path/to/logger"
# arguments the logger should be started with
Args = ["-config", "/path/to/config-file"]
\end{lstlisting}
\caption{Example plugin manager configuration file}
\end{figure}

\subsection{Honeypot Plugins}

\subsubsection{SSH}

\begin{figure}[H]
\begin{lstlisting}[language=Python,frame=single]
# address fake SSH server should listen on
Address = "localhost"
# port fake SSH server should listen on
Port = 8022
# full path to SSH Host private key that should be used
Key = "/path/to/private/key"
\end{lstlisting}
\caption{Example SSH plugin configuration file}
\end{figure}

\subsubsection{HTTP/HTTPS}

\begin{figure}[H]
\begin{lstlisting}[language=Python,frame=single]
# address the HTTP/HTTPS server should listen on
Host = "localhost"
# full path to TLS cert to use
Cert = "tls/dummy_cert.pem"
# full path to TLS private key to use
Key = "tls/dummy_key.pem"
# HTTP port to listen on
HTTPPort = 8080
# HTTPS port to listen on
HTTPSPort = 8443
# full path to template used for login page
LoginTemplate = "/path/to/templates/login.html"
\end{lstlisting}
\caption{Example HTTP/HTTPS plugin configuration file}
\end{figure}

\subsubsection{DNP3}

\begin{figure}[H]
\begin{lstlisting}[language=Python,frame=single]
# address DNP3 plugin should listen on
Address = "localhost"
# port plugin should listen on for incoming DMP3 connections
Port = 20000
\end{lstlisting}
\caption{Example DNP3 plugin configuration file}
\end{figure}

\subsection{Logging Plugins}

\subsubsection{Splunk}

\begin{figure}[H]
\begin{lstlisting}[language=Python,frame=single]
# each remote logging destination should be in an Endpoint block
[[Endpoint]]
# the remote endpoint host
Host = "localhost"
# the remote endpoint event collector point
Port = 8088
# the remote endpoint event collector url
URL = "/services/collector"
# auth token to be included with each message
AuthToken = "12345"
# OPTIONAL: if included, use this set of root CAs instead of the system
#			default. useful when using self-signed TLS certificates
RootCAs = ["/path/to/ca/cert.pem"]
\end{lstlisting}
\caption{Example Splunk logger configuration file}
\end{figure}
