\chapter{Requirements}

\section{Overall Project Requirements}

The project we received from Alliant Energy was to create a low interaction SCADA honeypot system for 28 of their power plants.  The system requirements for the device were as follows:

\begin{itemize}
    \item Capable of interfacing with SSH, HTTP and HTTPS
    \item Have a fake login page with key logger to record login attempts 
    \item Record logs of all connection attempts and send an alert to essential personnel upon irregularities such as increased ICMP traffic, port scans, and repeated attempts to connect to URLs.
    \item Rule set must be able to be fine-tuned to accommodate new/changed rules.
    \item Potential for a small intrusion detection system.
    \item Must be a low maintenance standalone device.
    \item Must be low power.
    \item Each of the 28 substations must be equipped with one.
\end{itemize}


\section{Proposed Solution}

To house the necessary software to meet these expectations we have decided to create a standalone system using a Raspberry Pi
microcontroller.  The Raspberry Pi is our preferred choice of hardware for the time being since it is a small Linux based
microcontroller that is low in cost and power requirements.  However, we would like to have multiple network interface cards in
order to have one port for implementing SSH and HTTP/S and one for listening on the network.

To accommodate Alliant’s need for SSH as well as HTTP services we have decided the best course of action would be to set up an
SSH server as well as a web server on the hardware device.  The SSH server will serve as a false gateway to inside the SCADA 
system.  The illegitimate user will be able to enter in a username and password at which time the TCP handshake will be terminated
and a log will be filed with the relevant information and passed on to a network administrator for handling.  The web server will
offer similar services for HTTP and HTTPS.  We plan to implement a fake login page visible to the unknown user.  At this time the
login page will always return invalid credentials and will be protecting nothing.  Later, if we wish to make the honeypot more
interactive we could make the username and password vulnerable to cracking methods and have it protect false data to see how the
user is tampering with it.  This information would be gathered and reported to the network administrator via an outside Splunk
server in Alliant’s datacenter.  The benefits of using a SPLUNK server is that it is a tested and proven utility that performs
the exact function necessary to efficiently alert the necessary employees.  The Splunk services are also configurable to capture
logs of ICMP ping sweeps, port scans and unwanted URL attempts.   

\subsection{Assessment of Proposed Solution}

The Raspberry Pi only comes with one interface card so we are currently looking at alternatives such as a USB interface or a different microcontroller.
Raspberry Pi’s also provide a limited amount of RAM which presents a potential problem for the implementation of a complex
intrusion detection system.  For this reason we are keeping the Raspberry Pi as our number one option for the time being but
continue to research other potential options.

\section{Validation and Acceptance Testing}

The testing requirements for each project requirement are as follows.
\begin{itemize}
\item SSH, HTTP, HTTPS - Create test scripts in GoLang to emulate several clients in addition to end to end testing.
\item Login - Observe successful end to end testing of Splunk parsing. Attempt site compromising attacks such as cross site scripting.
\item Logging - Attempt a series of port scans through NMAP. Attempt a series of connections through a test script.
\item Rule Set - Try all permutations of potential rule sets and test them.
\item IDS - Run all services with IDS and ensure the device is still functional.
\item Device - Check system requirements with client specifications. 
\item Power - Test device power consumption and ensure specifications match client needs.
\item Equipment - Monitor each device to ensure consistent uptime.
\end{itemize}
