\chapter{Interface and System Description}

\section{Technical Approach}

Our design and implementation strategy is guided by a handful of core
technical principles.

\subsubsection{Pluggable Architecture}

To accommodate evolving client needs and future maintainability and utility,
our core deliverable will consist of a plug-in architecture
for micro-services and logging modules, along with a default set of honeypot
plug-ins and a Splunk logging module. By deploying a flexible plug-in
architecture, additional logging back-ends and honeypot protocols can be
integrated easily into the full system.

\subsubsection{Automation}

Since our goal is to deploy \textit{many} devices at different
locations. We aim to automate as much of the build and configuration process
as possible. Therefore, we will use Ansible
\footnote{\url{http://www.ansible.com/}} for easy remote setup for
multiple devices. We will also use Docker
\footnote{\url{https://www.docker.com/}} to configure the individual
honeypot microservices. This setup will allow for simple and flexible device
rollouts and updates.

\subsubsection{Safe Languages}

The device will only deploy honeypot services written in a
safe\footnote{Used informally here to mean languages that enforce type
safety, memory safety, and provide built-in or standard library mechanisms to
write safe concurrent code.} language.

The core plugin architecture and the default plugins themselves will be written
in Go \footnote{\url{https://golang.org/}}.

\subsubsection{Honeypots Don't Get Real Capabilities}

Unlike medium- and high-interaction honeypots such as Kippo
\footnote{\url{https://github.com/desaster/kippo}}, our default honeypot
microservices will not provide any real system capabilities. Since the device will be
running in an internal, protected environment, \textit{any} traffic should
be considered an attack, and by only mimicking protocol initiation handshakes
without providing actual program capabilities, we gain some protection
against unknown exploits.

\subsubsection{Heavily Restrict Traffic}

The device will heavily restrict both ingress and egress traffic to a set of
known, trusted servers. This will help mitigate any damage a compromised
microservice on the device can cause.

\subsubsection{Disguise Administration Interface}

Since we aim to both trick an attacker into believing the services running
on the device are legitimate, as well as restrict access to \textit{functional}
services on the device, the administration interface will only be
accessible using Single Packet Authorization (SPA) 
\footnote{\url{http://www.blackhat.com/presentations/bh-usa-05/bh-us-05-madhat.pdf}}.

\section{Design Specification}

The system has 2 primary layers of interfaces:

\begin{enumerate}
    \item The External Interface, including components exposed to attackers and administrators as well as the alert logging interface
    \item The Internal Component Interface, through which honeypot services communicate with each other
\end{enumerate}

Each layer has a particular interface for each component. See Figure
\ref{figure:device} for a simplified diagram of the major device components.

\subsection{Layer 1: The External Interface}

The outermost layer of the device, the external interface, is the only point at
which \textit{incoming} network traffic is accepted. Access is controlled
by an IPTables\footnote{\url{http://linux.die.net/man/8/iptables}}
ruleset.

\subsubsection{The Public Interface}

The public interface, visible to anyone on the device's network, is the
interface through which attackers interact with the various honeypot
micro-sevices. Since the micro-services are designed to run as unprivileged,
isolated processes, the honeypots themselves run on high, non-standard
ports (e.g. 8022 for SSH and 8080 for HTTP).

Firewall rules forward traffic from the open, standard port (e.g. 22 for SSH)
to the high, non-standard port on local host where the actual honeypot
is listening.

\subsubsection{The Administration Interface}

Devices must be updated and individually configurable, so there needs to be a
way for administrators (or automated processes) to log in and perform routine 
maintenance. However, the device also should not expose a public SSH service
with real login capabilities, and the honeypot SSH service can't be used for
real system tasks.

Therefore, the device will use Single Packet Authorization
to disguise the administration interface, only opening the port for a single
connection to a particular internal IP address when presented with valid
cryptographic credentials.

\subsubsection{The Promiscuous Interface}

The device will run an IDS in order to pick up anomalies and attacks on
the greater internal network. The IDS will watch both incoming traffic to
the device as well as sniff traffic on the internal network through a
promiscuous interface.

\begin{figure}
\centering
{\scalefont{0.65}
% Graphic for TeX using PGF
% Title: /home/nskinkel/src/may1601/docs/fall2015/project_plan/report/Diagram1.dia
% Creator: Dia v0.97.3
% CreationDate: Sat Oct  3 18:51:05 2015
% For: nskinkel
% \usepackage{tikz}
% The following commands are not supported in PSTricks at present
% We define them conditionally, so when they are implemented,
% this pgf file will use them.
\ifx\du\undefined
  \newlength{\du}
\fi
\setlength{\du}{15\unitlength}
\begin{tikzpicture}
\pgftransformxscale{0.600000}
\pgftransformyscale{-0.600000}
\definecolor{dialinecolor}{rgb}{0.000000, 0.000000, 0.000000}
\pgfsetstrokecolor{dialinecolor}
\definecolor{dialinecolor}{rgb}{1.000000, 1.000000, 1.000000}
\pgfsetfillcolor{dialinecolor}
\definecolor{dialinecolor}{rgb}{1.000000, 1.000000, 1.000000}
\pgfsetfillcolor{dialinecolor}
\fill (5.650000\du,3.300000\du)--(5.650000\du,25.950000\du)--(41.750000\du,25.950000\du)--(41.750000\du,3.300000\du)--cycle;
\pgfsetlinewidth{0.100000\du}
\pgfsetdash{}{0pt}
\pgfsetdash{}{0pt}
\pgfsetmiterjoin
\definecolor{dialinecolor}{rgb}{0.000000, 0.000000, 0.000000}
\pgfsetstrokecolor{dialinecolor}
\draw (5.650000\du,3.300000\du)--(5.650000\du,25.950000\du)--(41.750000\du,25.950000\du)--(41.750000\du,3.300000\du)--cycle;
% setfont left to latex
\definecolor{dialinecolor}{rgb}{0.000000, 0.000000, 0.000000}
\pgfsetstrokecolor{dialinecolor}
\node at (23.700000\du,14.820000\du){};
\definecolor{dialinecolor}{rgb}{1.000000, 1.000000, 1.000000}
\pgfsetfillcolor{dialinecolor}
\fill (9.925000\du,12.200000\du)--(9.925000\du,23.550000\du)--(16.500000\du,23.550000\du)--(16.500000\du,12.200000\du)--cycle;
\pgfsetlinewidth{0.100000\du}
\pgfsetdash{}{0pt}
\pgfsetdash{}{0pt}
\pgfsetmiterjoin
\definecolor{dialinecolor}{rgb}{0.000000, 0.000000, 0.000000}
\pgfsetstrokecolor{dialinecolor}
\draw (9.925000\du,12.200000\du)--(9.925000\du,23.550000\du)--(16.500000\du,23.550000\du)--(16.500000\du,12.200000\du)--cycle;
% setfont left to latex
\definecolor{dialinecolor}{rgb}{0.000000, 0.000000, 0.000000}
\pgfsetstrokecolor{dialinecolor}
\node at (13.212500\du,18.070000\du){};
\pgfsetlinewidth{0.200000\du}
\pgfsetdash{}{0pt}
\pgfsetdash{}{0pt}
\pgfsetbuttcap
{
\definecolor{dialinecolor}{rgb}{0.000000, 0.000000, 0.000000}
\pgfsetfillcolor{dialinecolor}
% was here!!!
\pgfsetarrowsend{latex}
\definecolor{dialinecolor}{rgb}{0.000000, 0.000000, 0.000000}
\pgfsetstrokecolor{dialinecolor}
\draw (2.000000\du,3.400000\du)--(1.991667\du,25.770833\du);
}
\pgfsetlinewidth{0.200000\du}
\pgfsetdash{}{0pt}
\pgfsetdash{}{0pt}
\pgfsetbuttcap
{
\definecolor{dialinecolor}{rgb}{0.000000, 0.000000, 0.000000}
\pgfsetfillcolor{dialinecolor}
% was here!!!
\pgfsetarrowsend{latex}
\definecolor{dialinecolor}{rgb}{0.000000, 0.000000, 0.000000}
\pgfsetstrokecolor{dialinecolor}
\draw (2.991667\du,25.904167\du)--(3.000000\du,3.350000\du);
}
\definecolor{dialinecolor}{rgb}{1.000000, 1.000000, 1.000000}
\pgfsetfillcolor{dialinecolor}
\fill (10.100000\du,4.200000\du)--(10.100000\du,7.650000\du)--(17.050000\du,7.650000\du)--(17.050000\du,4.200000\du)--cycle;
\pgfsetlinewidth{0.100000\du}
\pgfsetdash{}{0pt}
\pgfsetdash{}{0pt}
\pgfsetmiterjoin
\definecolor{dialinecolor}{rgb}{0.000000, 0.000000, 0.000000}
\pgfsetstrokecolor{dialinecolor}
\draw (10.100000\du,4.200000\du)--(10.100000\du,7.650000\du)--(17.050000\du,7.650000\du)--(17.050000\du,4.200000\du)--cycle;
% setfont left to latex
\definecolor{dialinecolor}{rgb}{0.000000, 0.000000, 0.000000}
\pgfsetstrokecolor{dialinecolor}
\node at (13.575000\du,6.120000\du){Bro IDS};
\pgfsetlinewidth{0.100000\du}
\pgfsetdash{}{0pt}
\pgfsetdash{}{0pt}
\pgfsetbuttcap
{
\definecolor{dialinecolor}{rgb}{0.000000, 0.000000, 0.000000}
\pgfsetfillcolor{dialinecolor}
% was here!!!
\pgfsetarrowsend{latex}
\definecolor{dialinecolor}{rgb}{0.000000, 0.000000, 0.000000}
\pgfsetstrokecolor{dialinecolor}
\draw (2.950000\du,6.000000\du)--(10.051059\du,5.949875\du);
}
\definecolor{dialinecolor}{rgb}{1.000000, 1.000000, 1.000000}
\pgfsetfillcolor{dialinecolor}
\fill (4.533333\du,6.305000\du)--(4.533333\du,10.250000\du)--(6.765833\du,10.250000\du)--(6.765833\du,6.305000\du)--cycle;
% setfont left to latex
\definecolor{dialinecolor}{rgb}{0.000000, 0.000000, 0.000000}
\pgfsetstrokecolor{dialinecolor}
\node[anchor=west] at (4.533333\du,6.900000\du){};
% setfont left to latex
\definecolor{dialinecolor}{rgb}{0.000000, 0.000000, 0.000000}
\pgfsetstrokecolor{dialinecolor}
\node[anchor=west] at (4.533333\du,7.700000\du){Sniffed};
% setfont left to latex
\definecolor{dialinecolor}{rgb}{0.000000, 0.000000, 0.000000}
\pgfsetstrokecolor{dialinecolor}
\node[anchor=west] at (4.533333\du,8.500000\du){Traffic};
% setfont left to latex
\definecolor{dialinecolor}{rgb}{0.000000, 0.000000, 0.000000}
\pgfsetstrokecolor{dialinecolor}
\node[anchor=west] at (4.533333\du,9.300000\du){};
% setfont left to latex
\definecolor{dialinecolor}{rgb}{0.000000, 0.000000, 0.000000}
\pgfsetstrokecolor{dialinecolor}
\node[anchor=west] at (4.533333\du,10.100000\du){};
\definecolor{dialinecolor}{rgb}{1.000000, 1.000000, 1.000000}
\pgfsetfillcolor{dialinecolor}
\fill (22.466667\du,11.591667\du)--(22.466667\du,14.291667\du)--(28.066667\du,14.291667\du)--(28.066667\du,11.591667\du)--cycle;
\pgfsetlinewidth{0.100000\du}
\pgfsetdash{}{0pt}
\pgfsetdash{}{0pt}
\pgfsetmiterjoin
\definecolor{dialinecolor}{rgb}{0.000000, 0.000000, 0.000000}
\pgfsetstrokecolor{dialinecolor}
\draw (22.466667\du,11.591667\du)--(22.466667\du,14.291667\du)--(28.066667\du,14.291667\du)--(28.066667\du,11.591667\du)--cycle;
% setfont left to latex
\definecolor{dialinecolor}{rgb}{0.000000, 0.000000, 0.000000}
\pgfsetstrokecolor{dialinecolor}
\node at (25.266667\du,12.736667\du){Microservice};
% setfont left to latex
\definecolor{dialinecolor}{rgb}{0.000000, 0.000000, 0.000000}
\pgfsetstrokecolor{dialinecolor}
\node at (25.266667\du,13.536667\du){Honeypot};
\definecolor{dialinecolor}{rgb}{1.000000, 1.000000, 1.000000}
\pgfsetfillcolor{dialinecolor}
\fill (22.555000\du,16.541667\du)--(22.555000\du,19.241667\du)--(28.155000\du,19.241667\du)--(28.155000\du,16.541667\du)--cycle;
\pgfsetlinewidth{0.100000\du}
\pgfsetdash{}{0pt}
\pgfsetdash{}{0pt}
\pgfsetmiterjoin
\definecolor{dialinecolor}{rgb}{0.000000, 0.000000, 0.000000}
\pgfsetstrokecolor{dialinecolor}
\draw (22.555000\du,16.541667\du)--(22.555000\du,19.241667\du)--(28.155000\du,19.241667\du)--(28.155000\du,16.541667\du)--cycle;
% setfont left to latex
\definecolor{dialinecolor}{rgb}{0.000000, 0.000000, 0.000000}
\pgfsetstrokecolor{dialinecolor}
\node at (25.355000\du,17.686667\du){Microservice};
% setfont left to latex
\definecolor{dialinecolor}{rgb}{0.000000, 0.000000, 0.000000}
\pgfsetstrokecolor{dialinecolor}
\node at (25.355000\du,18.486667\du){Honeypot};
\definecolor{dialinecolor}{rgb}{1.000000, 1.000000, 1.000000}
\pgfsetfillcolor{dialinecolor}
\fill (22.476667\du,21.450000\du)--(22.476667\du,24.150000\du)--(28.076667\du,24.150000\du)--(28.076667\du,21.450000\du)--cycle;
\pgfsetlinewidth{0.100000\du}
\pgfsetdash{}{0pt}
\pgfsetdash{}{0pt}
\pgfsetmiterjoin
\definecolor{dialinecolor}{rgb}{0.000000, 0.000000, 0.000000}
\pgfsetstrokecolor{dialinecolor}
\draw (22.476667\du,21.450000\du)--(22.476667\du,24.150000\du)--(28.076667\du,24.150000\du)--(28.076667\du,21.450000\du)--cycle;
% setfont left to latex
\definecolor{dialinecolor}{rgb}{0.000000, 0.000000, 0.000000}
\pgfsetstrokecolor{dialinecolor}
\node at (25.276667\du,22.595000\du){Microservice};
% setfont left to latex
\definecolor{dialinecolor}{rgb}{0.000000, 0.000000, 0.000000}
\pgfsetstrokecolor{dialinecolor}
\node at (25.276667\du,23.395000\du){Honeypot};
\pgfsetlinewidth{0.100000\du}
\pgfsetdash{}{0pt}
\pgfsetdash{}{0pt}
\pgfsetmiterjoin
\pgfsetbuttcap
{
\definecolor{dialinecolor}{rgb}{0.000000, 0.000000, 0.000000}
\pgfsetfillcolor{dialinecolor}
% was here!!!
\pgfsetarrowsend{latex}
{\pgfsetcornersarced{\pgfpoint{0.000000\du}{0.000000\du}}\definecolor{dialinecolor}{rgb}{0.000000, 0.000000, 0.000000}
\pgfsetstrokecolor{dialinecolor}
\draw (16.550407\du,17.875000\du)--(19.483363\du,17.875000\du)--(19.483363\du,12.941667\du)--(22.416319\du,12.941667\du);
}}
\pgfsetlinewidth{0.100000\du}
\pgfsetdash{}{0pt}
\pgfsetdash{}{0pt}
\pgfsetmiterjoin
\pgfsetbuttcap
{
\definecolor{dialinecolor}{rgb}{0.000000, 0.000000, 0.000000}
\pgfsetfillcolor{dialinecolor}
% was here!!!
\pgfsetarrowsend{latex}
{\pgfsetcornersarced{\pgfpoint{0.000000\du}{0.000000\du}}\definecolor{dialinecolor}{rgb}{0.000000, 0.000000, 0.000000}
\pgfsetstrokecolor{dialinecolor}
\draw (16.550407\du,17.875000\du)--(19.488363\du,17.875000\du)--(19.488363\du,22.800000\du)--(22.426319\du,22.800000\du);
}}
\pgfsetlinewidth{0.100000\du}
\pgfsetdash{}{0pt}
\pgfsetdash{}{0pt}
\pgfsetmiterjoin
\pgfsetbuttcap
{
\definecolor{dialinecolor}{rgb}{0.000000, 0.000000, 0.000000}
\pgfsetfillcolor{dialinecolor}
% was here!!!
\pgfsetarrowsend{latex}
{\pgfsetcornersarced{\pgfpoint{0.000000\du}{0.000000\du}}\definecolor{dialinecolor}{rgb}{0.000000, 0.000000, 0.000000}
\pgfsetstrokecolor{dialinecolor}
\draw (16.550407\du,17.875000\du)--(19.527530\du,17.875000\du)--(19.527530\du,17.891667\du)--(22.504652\du,17.891667\du);
}}
% setfont left to latex
\definecolor{dialinecolor}{rgb}{0.000000, 0.000000, 0.000000}
\pgfsetstrokecolor{dialinecolor}
\node[anchor=west] at (23.700000\du,14.625000\du){};
\definecolor{dialinecolor}{rgb}{1.000000, 1.000000, 1.000000}
\pgfsetfillcolor{dialinecolor}
\fill (34.250000\du,8.200000\du)--(34.250000\du,20.150000\du)--(40.500000\du,20.150000\du)--(40.500000\du,8.200000\du)--cycle;
\pgfsetlinewidth{0.100000\du}
\pgfsetdash{}{0pt}
\pgfsetdash{}{0pt}
\pgfsetmiterjoin
\definecolor{dialinecolor}{rgb}{0.000000, 0.000000, 0.000000}
\pgfsetstrokecolor{dialinecolor}
\draw (34.250000\du,8.200000\du)--(34.250000\du,20.150000\du)--(40.500000\du,20.150000\du)--(40.500000\du,8.200000\du)--cycle;
% setfont left to latex
\definecolor{dialinecolor}{rgb}{0.000000, 0.000000, 0.000000}
\pgfsetstrokecolor{dialinecolor}
\node at (37.375000\du,14.370000\du){Logger};
\pgfsetlinewidth{0.100000\du}
\pgfsetdash{}{0pt}
\pgfsetdash{}{0pt}
\pgfsetmiterjoin
\pgfsetbuttcap
{
\definecolor{dialinecolor}{rgb}{0.000000, 0.000000, 0.000000}
\pgfsetfillcolor{dialinecolor}
% was here!!!
\pgfsetarrowsend{latex}
{\pgfsetcornersarced{\pgfpoint{0.000000\du}{0.000000\du}}\definecolor{dialinecolor}{rgb}{0.000000, 0.000000, 0.000000}
\pgfsetstrokecolor{dialinecolor}
\draw (17.099456\du,5.925000\du)--(31.550000\du,5.925000\du)--(31.550000\du,14.175000\du)--(34.200119\du,14.175000\du);
}}
\pgfsetlinewidth{0.100000\du}
\pgfsetdash{}{0pt}
\pgfsetdash{}{0pt}
\pgfsetmiterjoin
\pgfsetbuttcap
{
\definecolor{dialinecolor}{rgb}{0.000000, 0.000000, 0.000000}
\pgfsetfillcolor{dialinecolor}
% was here!!!
\pgfsetarrowsend{latex}
{\pgfsetcornersarced{\pgfpoint{0.000000\du}{0.000000\du}}\definecolor{dialinecolor}{rgb}{0.000000, 0.000000, 0.000000}
\pgfsetstrokecolor{dialinecolor}
\draw (28.115336\du,12.941667\du)--(31.550000\du,12.941667\du)--(31.550000\du,14.175000\du)--(34.200119\du,14.175000\du);
}}
\pgfsetlinewidth{0.100000\du}
\pgfsetdash{}{0pt}
\pgfsetdash{}{0pt}
\pgfsetmiterjoin
\pgfsetbuttcap
{
\definecolor{dialinecolor}{rgb}{0.000000, 0.000000, 0.000000}
\pgfsetfillcolor{dialinecolor}
% was here!!!
\pgfsetarrowsend{latex}
{\pgfsetcornersarced{\pgfpoint{0.000000\du}{0.000000\du}}\definecolor{dialinecolor}{rgb}{0.000000, 0.000000, 0.000000}
\pgfsetstrokecolor{dialinecolor}
\draw (28.205214\du,17.891667\du)--(31.550000\du,17.891667\du)--(31.550000\du,14.175000\du)--(34.200119\du,14.175000\du);
}}
\pgfsetlinewidth{0.100000\du}
\pgfsetdash{}{0pt}
\pgfsetdash{}{0pt}
\pgfsetmiterjoin
\pgfsetbuttcap
{
\definecolor{dialinecolor}{rgb}{0.000000, 0.000000, 0.000000}
\pgfsetfillcolor{dialinecolor}
% was here!!!
\pgfsetarrowsend{latex}
{\pgfsetcornersarced{\pgfpoint{0.000000\du}{0.000000\du}}\definecolor{dialinecolor}{rgb}{0.000000, 0.000000, 0.000000}
\pgfsetstrokecolor{dialinecolor}
\draw (28.126163\du,22.800000\du)--(31.550000\du,22.800000\du)--(31.550000\du,14.175000\du)--(34.200119\du,14.175000\du);
}}
% setfont left to latex
\definecolor{dialinecolor}{rgb}{0.000000, 0.000000, 0.000000}
\pgfsetstrokecolor{dialinecolor}
\node[anchor=west] at (23.700000\du,14.625000\du){};
% setfont left to latex
\definecolor{dialinecolor}{rgb}{0.000000, 0.000000, 0.000000}
\pgfsetstrokecolor{dialinecolor}
\node[anchor=west] at (20.150000\du,4.933333\du){Alerts aggregated and formatted for Splunk};
% setfont left to latex
\definecolor{dialinecolor}{rgb}{0.000000, 0.000000, 0.000000}
\pgfsetstrokecolor{dialinecolor}
\node[anchor=west] at (12.716667\du,10.700000\du){Honeypot traffic forwarded};
% setfont left to latex
\definecolor{dialinecolor}{rgb}{0.000000, 0.000000, 0.000000}
\pgfsetstrokecolor{dialinecolor}
\node[anchor=west] at (12.716667\du,11.500000\du){    to localhost services};
% setfont left to latex
\definecolor{dialinecolor}{rgb}{0.000000, 0.000000, 0.000000}
\pgfsetstrokecolor{dialinecolor}
\node[anchor=west] at (10.858333\du,16.637500\du){};
% setfont left to latex
\definecolor{dialinecolor}{rgb}{0.000000, 0.000000, 0.000000}
\pgfsetstrokecolor{dialinecolor}
\node[anchor=west] at (10.858333\du,17.437500\du){  Public};
% setfont left to latex
\definecolor{dialinecolor}{rgb}{0.000000, 0.000000, 0.000000}
\pgfsetstrokecolor{dialinecolor}
\node[anchor=west] at (10.858333\du,18.237500\du){Interface};
% setfont left to latex
\definecolor{dialinecolor}{rgb}{0.000000, 0.000000, 0.000000}
\pgfsetstrokecolor{dialinecolor}
\node[anchor=west] at (10.858333\du,19.037500\du){};
\pgfsetlinewidth{0.100000\du}
\pgfsetdash{}{0pt}
\pgfsetdash{}{0pt}
\pgfsetbuttcap
{
\definecolor{dialinecolor}{rgb}{0.000000, 0.000000, 0.000000}
\pgfsetfillcolor{dialinecolor}
% was here!!!
\pgfsetarrowsend{latex}
\definecolor{dialinecolor}{rgb}{0.000000, 0.000000, 0.000000}
\pgfsetstrokecolor{dialinecolor}
\draw (2.991667\du,17.837500\du)--(9.874697\du,17.862754\du);
}
\definecolor{dialinecolor}{rgb}{1.000000, 1.000000, 1.000000}
\pgfsetfillcolor{dialinecolor}
\fill (4.125000\du,17.909167\du)--(4.125000\du,21.054167\du)--(7.077500\du,21.054167\du)--(7.077500\du,17.909167\du)--cycle;
% setfont left to latex
\definecolor{dialinecolor}{rgb}{0.000000, 0.000000, 0.000000}
\pgfsetstrokecolor{dialinecolor}
\node[anchor=west] at (4.125000\du,18.504167\du){};
% setfont left to latex
\definecolor{dialinecolor}{rgb}{0.000000, 0.000000, 0.000000}
\pgfsetstrokecolor{dialinecolor}
\node[anchor=west] at (4.125000\du,19.304167\du){Incoming};
% setfont left to latex
\definecolor{dialinecolor}{rgb}{0.000000, 0.000000, 0.000000}
\pgfsetstrokecolor{dialinecolor}
\node[anchor=west] at (4.125000\du,20.104167\du){  Traffic};
% setfont left to latex
\definecolor{dialinecolor}{rgb}{0.000000, 0.000000, 0.000000}
\pgfsetstrokecolor{dialinecolor}
\node[anchor=west] at (4.125000\du,20.904167\du){};
\pgfsetlinewidth{0.100000\du}
\pgfsetdash{}{0pt}
\pgfsetdash{}{0pt}
\pgfsetmiterjoin
\pgfsetbuttcap
{
\definecolor{dialinecolor}{rgb}{0.000000, 0.000000, 0.000000}
\pgfsetfillcolor{dialinecolor}
% was here!!!
\pgfsetarrowsend{latex}
{\pgfsetcornersarced{\pgfpoint{0.000000\du}{0.000000\du}}\definecolor{dialinecolor}{rgb}{0.000000, 0.000000, 0.000000}
\pgfsetstrokecolor{dialinecolor}
\draw (6.433182\du,17.850127\du)--(6.433182\du,9.770833\du)--(13.575000\du,9.770833\du)--(13.575000\du,7.700036\du);
}}
% setfont left to latex
\definecolor{dialinecolor}{rgb}{0.000000, 0.000000, 0.000000}
\pgfsetstrokecolor{dialinecolor}
\node[anchor=west] at (1.258333\du,0.637500\du){};
% setfont left to latex
\definecolor{dialinecolor}{rgb}{0.000000, 0.000000, 0.000000}
\pgfsetstrokecolor{dialinecolor}
\node[anchor=west] at (1.258333\du,1.437500\du){Network};
% setfont left to latex
\definecolor{dialinecolor}{rgb}{0.000000, 0.000000, 0.000000}
\pgfsetstrokecolor{dialinecolor}
\node[anchor=west] at (1.258333\du,2.237500\du){ Traffic};
\pgfsetlinewidth{0.100000\du}
\pgfsetdash{}{0pt}
\pgfsetdash{}{0pt}
\pgfsetbuttcap
{
\definecolor{dialinecolor}{rgb}{0.000000, 0.000000, 0.000000}
\pgfsetfillcolor{dialinecolor}
% was here!!!
\pgfsetarrowsend{latex}
\definecolor{dialinecolor}{rgb}{0.000000, 0.000000, 0.000000}
\pgfsetstrokecolor{dialinecolor}
\draw (40.548572\du,14.173477\du)--(46.058333\du,14.170833\du);
}
\definecolor{dialinecolor}{rgb}{1.000000, 1.000000, 1.000000}
\pgfsetfillcolor{dialinecolor}
\fill (40.658333\du,11.575833\du)--(40.658333\du,13.920833\du)--(43.708333\du,13.920833\du)--(43.708333\du,11.575833\du)--cycle;
% setfont left to latex
\definecolor{dialinecolor}{rgb}{0.000000, 0.000000, 0.000000}
\pgfsetstrokecolor{dialinecolor}
\node[anchor=west] at (40.658333\du,12.170833\du){};
% setfont left to latex
\definecolor{dialinecolor}{rgb}{0.000000, 0.000000, 0.000000}
\pgfsetstrokecolor{dialinecolor}
\node[anchor=west] at (40.658333\du,12.970833\du){To Splunk};
% setfont left to latex
\definecolor{dialinecolor}{rgb}{0.000000, 0.000000, 0.000000}
\pgfsetstrokecolor{dialinecolor}
\node[anchor=west] at (40.658333\du,13.770833\du){};
\end{tikzpicture}

}
\caption{Simplified device internals}
\label{figure:device}
\end{figure}
 

\subsection{Layer 2: Low-Interaction Honeypot Micro-services}

We will provide a structured plug-in architecture that can run clearly defined
plug-ins for both arbitrary honeypot services as well as logging back-ends.
The plug-ins then run as isolated and unprivileged processes, using inter-process
communication over a well-defined protocol to communicate.

Each default honeypot micro-service plug-in set will be a low-interaction
honeypot, without full protocol capabilities for its emulated service and
with just enough functionality to discern:

\begin{itemize}
    \item If a passive attacker is accessing the device
    \item If an active attacker is accessing the device
    \item What, if any, credentials an active attacker attempts to use
    \item Metadata about the attacker (IP address, MAC address, date/time, etc.)
\end{itemize}

\subsubsection{Honeypot Plug-in Architecture}

Since the Go programming language does not support dynamic linking,
and thus compiling multiple plug-ins into a single executable becomes a messy
procedure, plug-ins themselves are implemented as independent programs
that each implement our structured plug-in interface. We define two types of
recognized plug-ins:
\begin{itemize}
    \item Honeypot micro-service plug-ins
    \item Logging plug-ins
\end{itemize}

The plug-in architecture has two important interfaces.

\paragraph{A Structured Public Plugin Interface}
The plugin architecture defines a small, public interface that all plugins
must implement. See Appendix \ref{appendix:honeypot-iface} for details of the
honeypot plugin interface and Appendix \ref{appendix:logger-iface} for details
of the logger plugin interface.

\paragraph{An IPC Mechanism}
Since plugins are isolated programs, they communicate using IPC (Inter-Process Communication). Specifically,
plugins use Go's net/rpc library to communicate. Note however that this
communication is handled transparently by the plugin architecture:
plugin implementers do not perform RPC (Remote Procedure Call) with other components. The
plugin controller itself performs RPC behind the scenes to coordinate
plugin execution and communication, and by implementing the standard
plugin interface, individual plugins get RPC behind the scenes for free.


\section{Design and Implementation Stages}

Design and implementation is broken up into three logical and somewhat
self-contained stages. Each stage encompasses a design component, an
implementation component, and an evaluation and testing component. Each stage
implements a particular subset of the desired functionality.

\subsection{Stage 1}

Design and Implementation Stage 1 will consist of a number of initial tasks,
design iterations, and prototyping. By completion of stage 1, the device
will have the core plugin architecture implemented, run a few small honeypot
microservices for common protocols, run a very minimal IDS rulset,
and have a minimal build automation procedure.

\subsubsection{Core Plugin Architecture}

The core plugin architecture, including all public interfaces and the RPC
and plugin registration mechanism, will be implemented in this early stage.

\subsubsection{Minimal Build Automation}

The device will be able to install all required dependencies automatically
as part of the deployment procedure.

\subsubsection{HTTP/HTTPS Honeypot}

The device will be running an HTTP/HTTPS honeypot with a fake login page.

\subsubsection{SSH Honeypot}

The device will be running an SSH honeypot.

\subsubsection{Pushing to Splunk}

Both the HTTP/HTTPS honeypot and the SSH honeypot should be able to push
alerts to an external Splunk server using a default Splunk logging plugin.

\subsubsection{Open High Ports}

At the client's request, the device will have 3 open high ports listening
for incoming traffic.

\subsubsection{Minimal IDS}

An IDS will be used to push alerts to Splunk if any incoming traffic is seen
on the 3 high ports.

\subsubsection{Unit Testing and Bug Fixes}

Each component of Stage 1 will have full unit tests (where applicable).


\subsection{Stage 2}

Design and Implementation Stage 2 will extend the work done in Stage 1
and provide the first SCADA protocol honeypot microservice plugin.

\subsubsection{Better Deployment Automation}

In Stage 2, general system hardening, custom firewall setup, and IDS
configuration will be automated as part of the deployment step.

\subsubsection{Promiscuous Network Interface}

A second network interface will be added to listen in promiscuous mode, used
by an IDS to pick up alerts from the greater internal network.

\subsubsection{Custom IDS Configuration}

The IDS will be configured to alert on domain-specific and network-specific
indicators, such as known energy-sector malware and internal ping sweeps.

\subsubsection{OPC Honeypot}

A honeypot microservice plugin will be implemented to emulate the OPC protocol,
the first SCADA protocol used on the device.

\subsubsection{Unit Testing and Bug Fixes}

Each component of Stage 2 will have full unit tests (where applicable).


\subsection{Stage 3}

Design and Implementation Stage 3 will finalize build automation, provide a
full integration testing suite, and allow time to add additional, custom
honeypot plugins to fit the client's partcular needs at different deployment
sites.

\subsubsection{Finalize Build and Deployment Automation}

Build and deployment automation procedures and configuration needs will be
finalized with the client.

\subsubsection{Single Packet Authorization}

Stage 3 will integrate a small Single Packet Authorization module into the
device for remote administration.

\subsubsection{Custom SCADA Protocols}

By working closely with the client, any additional SCADA protocols needed will
be implemented using the plugin microservice pattern established in design
stages 1 and 2.

\subsubsection{Unit Testing and Bug Fixes}

Each component of Stage 3 will have full unit tests (where applicable).

\subsubsection{Integration Testing}

A full integration test suite will be written to test data flow through the
system and, when possible, work to verify system correctness.

\section{Functional Testing Plan}

Each primary device componenet will have both unit tests and, when applicable,
integration tests. The device testing plan is specified
in Table \ref{table:test-plan}. Note that all microservice testing procedures
will follow the same strategy, so they have been collapsed to one table
entry.

\begin{table}[h]
\centering
\begin{tabular}{l  c  c}
Component & Unit Tests & Integration Tests \\
\hline
Plugin Architecture & \xmark & Tested with plugins \\
Build Automation & \xmark & Test deployment on virtual machines \\
IDS & \xmark & Script attacks and verify alerting \\
Microservices & \cmark & Attempt to login to active service \\
Logger & \cmark & Verify logging correct on Splunk instance \\
Single Packet Authorization & \cmark & Verify login succeeds/fails \\
Promiscuous Interface & \xmark & Verify IDS receives traffic \\
\end{tabular}
\caption{Functional Testing Plan}
\label{table:test-plan}
\end{table}
