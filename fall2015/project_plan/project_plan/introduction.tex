\chapter{Introduction}

\section{Project Introduction}

Cybersecurity is critical in maintaining the integrity and reliability of the electrical infrastructure in our society. It is of utmost importance to ensure that our electric grid is resilient since it is one of the most complex and critical infrastructure which every sector depends upon. Recently, the roles of the electrical power sector have shifted and now companies have a huge responsibility in ensuring continued security and resilience of the power grid. 

Attackers successfully compromised U.S. Department of Energy computer systems more than 150 times between 2010 and 2014. These numbers are expected to increase as computer systems begin to drive more of the power grid. According the Scott White, a professor of Homeland Security and Security management at Drexel University, "The potential for an adversary to disrupt, shut down (power systems), or worse ... is real here." As these threats become more persistent, so must the defense. Our project focuses on on this defense of twenty-eight of Alliant Energy's power substations, to ensure any potential attacker is stopped as quickly as possible. 

\section{Background}
With the dependency of electricity in the modern world, defending the functionality and integrity of electrical power plants is integral to the preservation and protection of our daily lives. Power plants handle extremely volatile resources on a continuous round-the-clock basis. In order to keep these systems up and running without fault they are monitored and controlled by numerous components on a SCADA(Supervisory Control and Data Acquisition) network. The integrity of this network is of vital importance. Normally these networks are almost completely isolated from the outside world. However, should an intruder somehow gain access to this network, it is important that IT personal be notified immediately so the cause of the intrusion be identified and closed as soon as possible.

An ongoing cyber-espionage campaign against targets in the energy sector has given attackers the ability to sabotage operations against their victims. These attackers, known as Dragonfly, have managed to compromise a number of important organizations in the energy industry. Dragonfly follows in the wake of Stuxnet, targeting industrial control systems (ICS). Dragonfly has used trojanized software to deliver malware to nearly 2,800 known victims thus far. Their aim, as far as it is known, has only been for espionage purposes. However, should they have used the sabotage capabilities that were made open to them, they could have caused damage and or large-scale energy disruption. This could have cost any of the many countries affected to lose millions of dollars in revenue. Because of such attacks, it is obvious that security for these kind of attempted breaches is decidedly necessary. 

This is where Honeypots come in to play. A honeypot is a dummy network node. To an attacker, the honeypot looks like a poorly protected network; in reality, this honeypot is a fake system that is isolated from the rest of the organization’s network and monitored closely by a security team. Honeypots disguise themselves as systems on the network, mimicking the behavior of other devices on the network while gathering information pertinent to identifying and nullifying the attack. Ideally, attackers on the network connect to the Honeypot and unknowingly provide information to the IT staff allowing them to interpret the source and content of the attack. The honeypot is able to catch specific types of network attacks directed specifically at certain SCADA network attacks.  However it is unable to catch everything. For this reason honeypots are often combined with an intrusion detection system (IDS).

Typically, an Intrusion Detection System is used to detect and alarm on suspected intrusions. This is done using signature-based, statistical anomaly based, or protocol analysis detection. These detection techniques allow IDS systems to analyze network traffic and detect if there is any abnormal traffic in the network such as ping sweeps, denial of service attacks, and viruses/worms etc. Any of these could potentially cause huge problems for a SCADA network because a honeypot alone would be unable to address these issues. Because of this it is advisable to combine the services of a honeypot and IDS to provide a layerd approach to network protection.

\section{Overview}
The system we had in mind to ensure the best possible security for Alliant Energy would be to use a Raspberry Pi running with two separate network interface cards (NICs). One NIC would facilitate all the inbound and outbound traffic for all of the network and SCADA protocols we want to monitor. The second NIC facilitates all traffic for the passive IDS system. This port only listens for abnormal network traffic and reports it to the on-board logging service before sending the desired information to proper network security administrators. By using this design we fit the initial design criteria by implementing a fully functioning honeypot and a small IDS system to catch common types of attacks while also keeping the device cheap and easy to maintain.

This system will be comprised of layered services. The outermost layer of the device is external interface and the only point at which incoming network traffic is accepted. Access is controlled by an IPTables rule set to provide some resistance to attackers to help conceal the honeypot. What will be visible is an interface that any attacker will be able to interact with. Should an attacker attempt to login or exploit any of the services available through the aforementioned interface, the system will send alerts to a logger and alert the necessary IT staff.

The architecture in this system will be a flexible plugin architecture. Each micro-service(HTTP, HTTPS, SSH, etc.) will function as a plugin, and communicate with the logger plugin. This will create maximum flexibility and allow the system to be highly extensible. This is ideal because should an admin want to add a new micro-service or logging functionality, all they need to do is write a small plugin and the system will be able to handle it and begin running it.

A honeypot's greatest value lies in its simplicity, it's a device that is intended to be compromised. This means that there is no production traffic going to or from the device. Any time a connection is made to the honeypot, it is most likely to be a probe, scan, or even attack. Any time a connection is initiated from the honeypot, this most likely means the honeypot was compromised. We believe our honeypot will prove a viable security product. With increasing threats to not only the electrical sector, but the general population as a whole. It is entirely necessary to do everything possible to reduce or eliminate risks to an organization's critical assets; this is what our product aims to do.
