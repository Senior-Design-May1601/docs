\chapter{Testing and Modeling}
Testing for this project can be done through a combination of unit testing and simulations. Four components are required to be tested. Ansible, SSH, WebAuth, and the Splunk Logger. Ansible will require simulations to determine if it can successfully install and configure multiple devices. SSH, WebAuth, and Splunk will too require simulations to evaluate their ability to handle several clients. In addition, these three components will require unit testing to check for edge cases and invalid inputs.

\section{Simulations and Modeling}
An open source program named Vagrant will be used for simulations. Vagrant allows for the quick creation of a virtual machine that can be used in this instance to create a simulated device. Ansible will take a newly created machine, run its' installations, and start all services. Multiple machines can be created by using a Vagrant file. Additional machines can be created in a similar method with the ability to make SSH, HTTP, and HTTPS connections. This will simulate several clients attempting to communicate with our Honeypot.

\section{Implementation Issues and Challenges}
The biggest challenge to testing is proper configuration of Vagrant and unit testing for inputs and edge cases. Writing appropriate test code for that will emulate client SSH, HTTP, and HTTPS calls will also be a challenge.

\section{Testing Procedures and Specifications}
The following list can be used as a procedural guideline for testing. Specifications can be described as all services performing their function correctly.
\begin{enumerate}
\item Install and configure Vagrant on the host machine. 
\item Run Vagrant file to start Vagrant on several virtual machines.
\item Start Ansible to install services on VM. Ansible configures and installs IP-Tables firewall, WebAuth, SSH, and Splunk logger
\item Start Vagrant on client VM's.
\item Simulate HTTP, HTTPS, and SSH calls. SSH and Http calls are logged to external Splunk server using JSON format. If there is an error, the Splunk logger will cache the event.
\item Evaluate Splunk Logs for JSON formatting
\item Use GoLang Test package to build test cases for each unit in WebAuth, Splunk, and SSH.
\item Run test cases
\item Evaluate results
\end{enumerate}
