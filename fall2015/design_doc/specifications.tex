\chapter{Specifications}

\section{I/O Specification}

The system I/O is broken down into three sections, each with
a particular set of inputs and outputs:

\begin{enumerate}
    \item Public Interface
    \item Network Traffic
    \item Administration
\end{enumerate}

Note that all references to ``Alert(s) to Splunk'' are output of the
form summarized in Figure \ref{figure:alert-format}, along with
context-specific details (for instance, in the case of an attempted SSH
login, the username and password supplied by the attacker) as detailed in
Appendix \ref{appendix:https-alert}.

\subsection{Public Interface}

The public interface is visible to normal actors on the internal network.
It is the only interface that attackers targetting the system will
\textit{directly} interact with (attackers may indirectly interact with
the system through sniffed network traffic). Each honeypot service running
on the system has a particular set of inputs defined by the underlying
protocol specification for the particular service, as well as dual
outputs: one output to the attacker (protocol-specific), and one output
to the internal logging system (same format for all services).

A summary of the protocol-specific inputs and outputs follows.

\begin{table}
\centering
\begin{tabular}{l l l}
Traffic Type & External Output & Internal Logged Output \\
\hline
SSH & Authentication Failed SSH Protocol Message & Alert to Splunk \\
Non-SSH & No Output & No Output \\
\end{tabular}
\caption{SSH Honeypot Outputs}
\label{table:ssh-output}
\end{table}

\begin{table}
\centering
\begin{tabular}{l l l l}
Traffic Type & Path & External Output & Internal Logged Output \\
\hline
HTTP & /login & Access Denied HTML Message & Alert to Splunk \\
HTTP & Any other path & HTML Login Page & Alert to Splunk \\
non-HTTP & N/A & No Output & No Output \\
\end{tabular}
\caption{HTTP Honeypot Outputs}
\label{table:http-output}
\end{table}

\begin{table}
\centering
\begin{tabular}{l l l l}
Traffic Type & Path & External Output & Internal Logged Output \\
\hline
HTTPS & /login & Access Denied HTML Message & Alert to Splunk \\
HTTPS & Any other path & HTML Login Page & Alert to Splunk \\
non-HTTPS & N/A & No Output & No Output \\
\end{tabular}
\caption{HTTPS Honeypot Outputs}
\label{table:https-output}
\end{table}

\begin{table}
\centering
\begin{tabular}{l l l l}
Traffic Type & External Output & Internal Logged Output \\
\hline
Valid SCADA Protocol & Access Denied Protocol Message & Alert to Splunk \\
Invalid SCADA Protocol & No Output & No Output \\
\end{tabular}
\caption{SCADA Honeypot Outputs}
\label{table:scada-output}
\end{table}

\subsubsection{SSH Honeypot I/O}

\paragraph{Input}

All TCP traffic on port 22 sent to the device. The SSH Honeypot ignores
non-SSH traffic on port 22.

\paragraph{Output}

The SSH Honeypot output is summarized in Table \ref{table:ssh-output}.

\subsubsection{HTTP Honeypot I/O}

\paragraph{Input}

All TCP traffic on port 80 sent to the device. The HTTP Honeypot ignores
non-HTTP traffic on port 80.

\paragraph{Output}

The HTTP Honeypot output is summarized in Table \ref{table:http-output}.

\subsubsection{HTTPS Honeypot I/O}

\paragraph{Input}

All TCP traffic on port 443 sent to the device. The HTTPS Honeypot ignores
non-HTTPS traffic on port 443.

\paragraph{Output}

The HTTPS Honeypot output is summarized in Table \ref{table:https-output}.

\subsubsection{SCADA Honeypot I/O}

This section summarizes the inputs and outputs for \textit{all} SCADA
honeypot services. Note that particular SCADA protocols, while existing on
the device as different services, will have logically equivalent I/O,
simply formatted to conform to the particular SCADA protocol in use. SCADA
honeypot I/O is thus summarized as a group.

\paragraph{Input}

Any SCADA traffic sent to the device on an open port.

\paragraph{Output}

The SCADA Honeypot service output is summarized in Table
\ref{table:scada-output}.


\subsection{Traffic}

Traffic inputs to the system originate from two sources:

\begin{enumerate}
    \item Sniffed network traffic
    \item High port traffic to device
\end{enumerate}

High port traffic to the device is traffic sent directly to an open port
on the device on which there is honeypot service running.

\subsubsection{Sniffed Network Traffic}

\paragraph{Input}

Sniffed network traffic is traffic observed by a network interface listening
in promiscuous mode. This input is any ethernet traffic on the same network
as the device.

\paragraph{Output}

In all cases, output is either none or an alert logged to Splunk. The decision
of whether or not to send an alert is made by the system IDS.

\subsection{Administration}

Device administration is done through SSH. In all cases, administration
traffic consists of SSH protocol messages.


\section{Interface Specifications}

The system deals with four primary interfaces:

\begin{enumerate}
    \item Administration Interface
    \item Public Interface
    \item Splunk Interface
    \item Promiscuous Network Interface
\end{enumerate}

A brief description of each primary interface follows.

\subsection{Administration Interface}

All administration is done over SSH. The administration interface is
simply an SSH server, listening on a \textit{different} network than
the public interface (an internal wireless network, for instance).  Administrators confirm access with each device through Single Packet Authentication to validate they have administrative rights to the device.

\subsection{Public Interface}

The public interface is the interface other users/hackers on the same network
as the device are presented with. This interface is a set of open ports,
one for each honeypot micro-service, and a handful of open high ports with
no services listening.

Since the device will be deployment to numerous (28) different locations,
and every device's configuration may vary slightly (i.e. a different set of
high ports open, or a different SCADA protocol running), an exhaustive
listing of open ports is not practical or possible at this design stage
(the list of open ports is configurable by the client).

However, every device follows the same pattern: one open port per honeypot
microservice, and a handful of open high ports.

\subsection{Splunk Interface}

The Splunk interface is the interface the device uses to push alerts to
administrators. The Splunk interface is interacted with through a
REST API, the details of which are specified by Splunk itself\footnote{\url{http://docs.splunk.com/Documentation/Splunk/latest/RESTREF/RESTprolog}}.

The alerts themselves are formatted JSON\footnote{\url{www.json.org}} using
format specified in Figure \ref{figure:alert-format}, displayed in pseudo-JSON
for convenience (format subject to change based on continuing feedback from
client).

Each "service type" entry is contains service-specific details (for instance,
an SSH alert may contain a username field, a password field, and a key field).
Additional data such as time of alert and originating device are handled
through metadata inherent in a Splunk REST API call.

See Appendix \ref{appendix:https-alert} for a full example of an alert from
the HTTPS microservice honeypot.

\begin{figure}
\begin{lstlisting}[language=Python,frame=single,showstringspaces=false]
{"event":{
    "source address": string
    "source port": int
    "service type": string
    "service type":{
        "service specific field A": type,
        "service specific field B": type,
        "service specific field N": type,
    }
}
\end{lstlisting}
\caption{General Alert JSON Format}
\label{figure:alert-format}
\end{figure}

\subsection{Promiscuous Network Interface}

The final interface is the promiscuous network interface. This is simply
a network interface listening in promiscous mode, and the system's IDS
will interpret the traffic and, if appropriate, log an appropriate alert
to Splunk using the REST API and an appropriate JSON format
\ref{figure:alert-format}.

\section{Hardware/Software Specifications}

\subsubsection{Hardware Specification}

The following components make up the hardware specification:

\begin{itemize}
    \item Raspberry Pi 2 Model B
    \item Quad-Core 900 MHz Processor
    \item 1 GB RAM
    \item 8 GB Micro SD Card
    \item 150 Mbps WiFi Adapter
    \item 2.5A USB Power Supply with Micro USB Cable
    \item CanaKit Raspberry Pi 2 Case
\end{itemize}

\subsubsection{Software Specification}

All system components implemented by this development team will be written
in the Go Programming Language\footnote{\url{https://golang.org/}}.

Table \ref{table:tools} summarizes the free software tools the system will use.

\begin{table}[h]
\centering
\begin{tabular}{l l l}
Tool Name & Purpose & Brief Description \\
\hline
Raspbian\tablefootnote{\url{https://www.raspbian.org/}} & Operating System & Debian GNU/Linux ARM port \\
Ansible\tablefootnote{\url{http://www.ansible.com/}} & Configuration and Setup & Build and deployment automation \\
Docker\tablefootnote{\url{https://www.docker.com/}} & Isolated microservice environment & Linux container wrapper \\
\end{tabular}
\caption{Free Software Tools Used}
\label{table:tools}
\end{table}
