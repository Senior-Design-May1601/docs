\chapter{System Level Design}

\section{System Requirements}
When requirements are gathered, it is important to break them up into "functional" requirements, and "non-functional" requirements. It is necessary to note the difference between the two, as they are key to understanding what to develop, and how to accomplish a project. Functional requirements are often thought of as a description of a system's behavior, and non-functional requirements elaborate performance characteristics of the system. In this section, a list of functional requirements, as well as non-functional requirements can be found.

\subsection{Functional Requirements:}
The Functional Requirements for the SCADA honeypot system are as follows.
\begin{itemize}
\item The system must interface with SSH, HTTP, and HTTPS protocols.
\item The system will use a fake login page with key logger to record login attempts.
\item The system will  have the ability to record logs of all connection attempts.
\item The system must contain a small intrusion detection system (IDS).
\item The system must send alerts to administrative personnel upon detection irregularities that include, but are not limited to, increased ICMP traffic, port scans, and repeated attempts to connect to URLs.
\item A rule set must be included that allows for fine-tuning to accommodate new or changed rules.
\end{itemize}

\subsection{Non-Functional Requirements:}
The Non-Functional Requirements for the SCADA honeypot system are as follows.
\begin{itemize}
\item The system must be low maintenance.
\item Must be a standalone device.
\item Must be low power.
\item Allow the support of no fewer than 28 devices.
\item Have an up-time of no less than 99 percent.
\end{itemize}

\section{Functional Decomposition}

The SCADA honeypot to be built for twenty-eight of Alliant Energy's powerplant substation will have seven main components that allow it to mimic and log potential attacks.  These components consist of a public firewall, an SSH server, a local webserver, an Ansible configuration manager, a log generator, a localized intrusion detection system, and a centralized Splunk server.  To gain a more complete understanding of how data is processed through each honeypot the next few paragraphs will describe the purpose and functionality of each component.

\subsection{Parts Breakdown:}

\subsubsection{Public Firewall}
The public firewall also known as the public interface is one of two means of ingress to the system.  The firewall setup uses Linux IP tables to control the flow of traffic to only information coming in on ports 22(SSH), 80(HTTP), 443(HTTPS) and port 2222.  These ports are opened up for the reason that they are the bare minimum required to run the services of the on-board servers.  This give the illusion that the honeypot is a secure service and helps to obscure its true intentions to a malicious user.  Note: The public firewall does not interact with the second network interface card because it is monitored directly by the Intrusion Detection System.

\subsubsection{SSH Server}
The SSH server, setup after the public facing firewall, is a mock server that only simulates the authentication process of the SSH handshake.  However the server setup on the honeypot is designed to terminate the handshake once the attempting user submits their credentials.  The SSH server then sends the username, password and other important information to the logger to be processed into one cohesive log. The user is then denied access after a short delay to mimic the background process of checking a database.

\subsubsection{Local Web-server}
The local web-server implemented in the design of the honeypot system works in a similar fashion to the SSH server. Essentially if a malicious user attempts to connect via HTTP/HTTPS they will be directed to the webserver.  This false web-server acts as an information gathering tool for Alliants admin.  Once inside the firewall the user will be presented with a login page that asks for a username and password.  Once entered the information is logged and sent to the logger just like the SSH server.  Upon entering their credential the system will wait for a period of time before displaying that the credentials entered are not accepted.  This provides the ability to log multiple attempts to access the system through HTTP/HTTPS.

\subsubsection{Ansible Configuration Manager}
The Ansible Configuration Manager is responsible for the setting up the initial configuration of each type of server and also pushing any patches that are necessary for the device to run properly.  Because of this it will only be accessible by network users with administrative rights. By using Ansible we are able to setup a playbook that allows us to configure our device with whatever protocols and services we want and then push all changes to every device at once, remotely.  This eliminates the need to physically go into every machine and make changes on a machine to machine bases.

\subsubsection{Log Generator}
The log generating component of the honeypot takes information from the all other components and formats them into an easy to read informative text file that can be sent to a central sever in Alliant Energy's data center.  These logs will consistently record normal network activity as well as any attacks that occur on the device.  The log format will be structured in accordance with Alliant Energy's choosing, but will contain information on the time the attack was initiated, what type of attack it was, the username and password, and other credentials.  This data will then be relayed back to a Splunk server at Alliant's main data center for analysis.

\subsubsection{Splunk Server}
The splunk server is a component in Alliant's data center that receives the logs generated in the honeypot and analyzes them to determine if an attack has happened according to Alliant's definition of an attack.  When an attack does occur the splunk server will contact an administrator via email to alert that an attack.  From this information the administrators can make  the appropriate response to keep the network secure.

\subsubsection{Bro Intrusion Detection System}
The Bro IDS software listens on a separate network interface card than the firewall and the other servers within the honeypot.  This second NIC is unregulated and connects directly to the IDS, meaning that is allows all traffic in.  By doing this the honeypot will be able to capture any other attack/malicious traffic that occur within the SCADA system.  However none of the other systems will be compromised as a result of this because each NIC is physically separate from each other and only connect at the logger which works the same way for either system.  

\section{System Analysis}
This project is routinely implemented in industry with high success rates. Several software implementations of SCADA honeypots already exist in the open source community. However, our implementation will be custom because the client has a secondary goal of including an IDS on the system. Creating the honeypot software from scratch will minimize the resources required from the platform. This is important because the computer needs to be a small standalone device. A Raspberry PI only has 1GB of RAM for example and an IDS typically uses a lot of memory and process cycles. Most risk can be mitigated with proper contingency planning. Installing an IDS on a small platform such as a Raspberry PI poses a challenge.  However we believe that we can compress the IDS to only include core functions Alliant wants to reduce the  resources required to run it.  The overall honeypot implementation carries a low risk of failure.

The overall plan to deal with unforeseen changes to the project is to run all software implementations in a plugin architecture on a standard raspberry pi with an additional network interface card added.  The raspberry pi was chosen for its' adaptability and we are confident if any changes are needed the pi will be able to handle them.  The flexible plugin architecture was chosen specifically because we cannot foresee all of the services Alliant might want to run in the future.  By making a plugin architecture with a default set of plugins that were specifically asked for in the project we make it so Alliant gets the services they want today, and the ability to easily update the devices with new services in the future.

\section{Block Diagrams}

\begin{figure}[h]
\centering
{\scalefont{0.70}
% Graphic for TeX using PGF
% Title: /home/nskinkel/src/may1601/docs/fall2015/project_plan/report/Diagram1.dia
% Creator: Dia v0.97.3
% CreationDate: Sat Oct  3 18:51:05 2015
% For: nskinkel
% \usepackage{tikz}
% The following commands are not supported in PSTricks at present
% We define them conditionally, so when they are implemented,
% this pgf file will use them.
\ifx\du\undefined
  \newlength{\du}
\fi
\setlength{\du}{15\unitlength}
\begin{tikzpicture}
\pgftransformxscale{0.600000}
\pgftransformyscale{-0.600000}
\definecolor{dialinecolor}{rgb}{0.000000, 0.000000, 0.000000}
\pgfsetstrokecolor{dialinecolor}
\definecolor{dialinecolor}{rgb}{1.000000, 1.000000, 1.000000}
\pgfsetfillcolor{dialinecolor}
\definecolor{dialinecolor}{rgb}{1.000000, 1.000000, 1.000000}
\pgfsetfillcolor{dialinecolor}
\fill (5.650000\du,3.300000\du)--(5.650000\du,25.950000\du)--(41.750000\du,25.950000\du)--(41.750000\du,3.300000\du)--cycle;
\pgfsetlinewidth{0.100000\du}
\pgfsetdash{}{0pt}
\pgfsetdash{}{0pt}
\pgfsetmiterjoin
\definecolor{dialinecolor}{rgb}{0.000000, 0.000000, 0.000000}
\pgfsetstrokecolor{dialinecolor}
\draw (5.650000\du,3.300000\du)--(5.650000\du,25.950000\du)--(41.750000\du,25.950000\du)--(41.750000\du,3.300000\du)--cycle;
% setfont left to latex
\definecolor{dialinecolor}{rgb}{0.000000, 0.000000, 0.000000}
\pgfsetstrokecolor{dialinecolor}
\node at (23.700000\du,14.820000\du){};
\definecolor{dialinecolor}{rgb}{1.000000, 1.000000, 1.000000}
\pgfsetfillcolor{dialinecolor}
\fill (9.925000\du,12.200000\du)--(9.925000\du,23.550000\du)--(16.500000\du,23.550000\du)--(16.500000\du,12.200000\du)--cycle;
\pgfsetlinewidth{0.100000\du}
\pgfsetdash{}{0pt}
\pgfsetdash{}{0pt}
\pgfsetmiterjoin
\definecolor{dialinecolor}{rgb}{0.000000, 0.000000, 0.000000}
\pgfsetstrokecolor{dialinecolor}
\draw (9.925000\du,12.200000\du)--(9.925000\du,23.550000\du)--(16.500000\du,23.550000\du)--(16.500000\du,12.200000\du)--cycle;
% setfont left to latex
\definecolor{dialinecolor}{rgb}{0.000000, 0.000000, 0.000000}
\pgfsetstrokecolor{dialinecolor}
\node at (13.212500\du,18.070000\du){};
\pgfsetlinewidth{0.200000\du}
\pgfsetdash{}{0pt}
\pgfsetdash{}{0pt}
\pgfsetbuttcap
{
\definecolor{dialinecolor}{rgb}{0.000000, 0.000000, 0.000000}
\pgfsetfillcolor{dialinecolor}
% was here!!!
\pgfsetarrowsend{latex}
\definecolor{dialinecolor}{rgb}{0.000000, 0.000000, 0.000000}
\pgfsetstrokecolor{dialinecolor}
\draw (2.000000\du,3.400000\du)--(1.991667\du,25.770833\du);
}
\pgfsetlinewidth{0.200000\du}
\pgfsetdash{}{0pt}
\pgfsetdash{}{0pt}
\pgfsetbuttcap
{
\definecolor{dialinecolor}{rgb}{0.000000, 0.000000, 0.000000}
\pgfsetfillcolor{dialinecolor}
% was here!!!
\pgfsetarrowsend{latex}
\definecolor{dialinecolor}{rgb}{0.000000, 0.000000, 0.000000}
\pgfsetstrokecolor{dialinecolor}
\draw (2.991667\du,25.904167\du)--(3.000000\du,3.350000\du);
}
\definecolor{dialinecolor}{rgb}{1.000000, 1.000000, 1.000000}
\pgfsetfillcolor{dialinecolor}
\fill (10.100000\du,4.200000\du)--(10.100000\du,7.650000\du)--(17.050000\du,7.650000\du)--(17.050000\du,4.200000\du)--cycle;
\pgfsetlinewidth{0.100000\du}
\pgfsetdash{}{0pt}
\pgfsetdash{}{0pt}
\pgfsetmiterjoin
\definecolor{dialinecolor}{rgb}{0.000000, 0.000000, 0.000000}
\pgfsetstrokecolor{dialinecolor}
\draw (10.100000\du,4.200000\du)--(10.100000\du,7.650000\du)--(17.050000\du,7.650000\du)--(17.050000\du,4.200000\du)--cycle;
% setfont left to latex
\definecolor{dialinecolor}{rgb}{0.000000, 0.000000, 0.000000}
\pgfsetstrokecolor{dialinecolor}
\node at (13.575000\du,6.120000\du){Bro IDS};
\pgfsetlinewidth{0.100000\du}
\pgfsetdash{}{0pt}
\pgfsetdash{}{0pt}
\pgfsetbuttcap
{
\definecolor{dialinecolor}{rgb}{0.000000, 0.000000, 0.000000}
\pgfsetfillcolor{dialinecolor}
% was here!!!
\pgfsetarrowsend{latex}
\definecolor{dialinecolor}{rgb}{0.000000, 0.000000, 0.000000}
\pgfsetstrokecolor{dialinecolor}
\draw (2.950000\du,6.000000\du)--(10.051059\du,5.949875\du);
}
\definecolor{dialinecolor}{rgb}{1.000000, 1.000000, 1.000000}
\pgfsetfillcolor{dialinecolor}
\fill (4.533333\du,6.305000\du)--(4.533333\du,10.250000\du)--(6.765833\du,10.250000\du)--(6.765833\du,6.305000\du)--cycle;
% setfont left to latex
\definecolor{dialinecolor}{rgb}{0.000000, 0.000000, 0.000000}
\pgfsetstrokecolor{dialinecolor}
\node[anchor=west] at (4.533333\du,6.900000\du){};
% setfont left to latex
\definecolor{dialinecolor}{rgb}{0.000000, 0.000000, 0.000000}
\pgfsetstrokecolor{dialinecolor}
\node[anchor=west] at (4.533333\du,7.700000\du){Sniffed};
% setfont left to latex
\definecolor{dialinecolor}{rgb}{0.000000, 0.000000, 0.000000}
\pgfsetstrokecolor{dialinecolor}
\node[anchor=west] at (4.533333\du,8.500000\du){Traffic};
% setfont left to latex
\definecolor{dialinecolor}{rgb}{0.000000, 0.000000, 0.000000}
\pgfsetstrokecolor{dialinecolor}
\node[anchor=west] at (4.533333\du,9.300000\du){};
% setfont left to latex
\definecolor{dialinecolor}{rgb}{0.000000, 0.000000, 0.000000}
\pgfsetstrokecolor{dialinecolor}
\node[anchor=west] at (4.533333\du,10.100000\du){};
\definecolor{dialinecolor}{rgb}{1.000000, 1.000000, 1.000000}
\pgfsetfillcolor{dialinecolor}
\fill (22.466667\du,11.591667\du)--(22.466667\du,14.291667\du)--(28.066667\du,14.291667\du)--(28.066667\du,11.591667\du)--cycle;
\pgfsetlinewidth{0.100000\du}
\pgfsetdash{}{0pt}
\pgfsetdash{}{0pt}
\pgfsetmiterjoin
\definecolor{dialinecolor}{rgb}{0.000000, 0.000000, 0.000000}
\pgfsetstrokecolor{dialinecolor}
\draw (22.466667\du,11.591667\du)--(22.466667\du,14.291667\du)--(28.066667\du,14.291667\du)--(28.066667\du,11.591667\du)--cycle;
% setfont left to latex
\definecolor{dialinecolor}{rgb}{0.000000, 0.000000, 0.000000}
\pgfsetstrokecolor{dialinecolor}
\node at (25.266667\du,12.736667\du){Microservice};
% setfont left to latex
\definecolor{dialinecolor}{rgb}{0.000000, 0.000000, 0.000000}
\pgfsetstrokecolor{dialinecolor}
\node at (25.266667\du,13.536667\du){Honeypot};
\definecolor{dialinecolor}{rgb}{1.000000, 1.000000, 1.000000}
\pgfsetfillcolor{dialinecolor}
\fill (22.555000\du,16.541667\du)--(22.555000\du,19.241667\du)--(28.155000\du,19.241667\du)--(28.155000\du,16.541667\du)--cycle;
\pgfsetlinewidth{0.100000\du}
\pgfsetdash{}{0pt}
\pgfsetdash{}{0pt}
\pgfsetmiterjoin
\definecolor{dialinecolor}{rgb}{0.000000, 0.000000, 0.000000}
\pgfsetstrokecolor{dialinecolor}
\draw (22.555000\du,16.541667\du)--(22.555000\du,19.241667\du)--(28.155000\du,19.241667\du)--(28.155000\du,16.541667\du)--cycle;
% setfont left to latex
\definecolor{dialinecolor}{rgb}{0.000000, 0.000000, 0.000000}
\pgfsetstrokecolor{dialinecolor}
\node at (25.355000\du,17.686667\du){Microservice};
% setfont left to latex
\definecolor{dialinecolor}{rgb}{0.000000, 0.000000, 0.000000}
\pgfsetstrokecolor{dialinecolor}
\node at (25.355000\du,18.486667\du){Honeypot};
\definecolor{dialinecolor}{rgb}{1.000000, 1.000000, 1.000000}
\pgfsetfillcolor{dialinecolor}
\fill (22.476667\du,21.450000\du)--(22.476667\du,24.150000\du)--(28.076667\du,24.150000\du)--(28.076667\du,21.450000\du)--cycle;
\pgfsetlinewidth{0.100000\du}
\pgfsetdash{}{0pt}
\pgfsetdash{}{0pt}
\pgfsetmiterjoin
\definecolor{dialinecolor}{rgb}{0.000000, 0.000000, 0.000000}
\pgfsetstrokecolor{dialinecolor}
\draw (22.476667\du,21.450000\du)--(22.476667\du,24.150000\du)--(28.076667\du,24.150000\du)--(28.076667\du,21.450000\du)--cycle;
% setfont left to latex
\definecolor{dialinecolor}{rgb}{0.000000, 0.000000, 0.000000}
\pgfsetstrokecolor{dialinecolor}
\node at (25.276667\du,22.595000\du){Microservice};
% setfont left to latex
\definecolor{dialinecolor}{rgb}{0.000000, 0.000000, 0.000000}
\pgfsetstrokecolor{dialinecolor}
\node at (25.276667\du,23.395000\du){Honeypot};
\pgfsetlinewidth{0.100000\du}
\pgfsetdash{}{0pt}
\pgfsetdash{}{0pt}
\pgfsetmiterjoin
\pgfsetbuttcap
{
\definecolor{dialinecolor}{rgb}{0.000000, 0.000000, 0.000000}
\pgfsetfillcolor{dialinecolor}
% was here!!!
\pgfsetarrowsend{latex}
{\pgfsetcornersarced{\pgfpoint{0.000000\du}{0.000000\du}}\definecolor{dialinecolor}{rgb}{0.000000, 0.000000, 0.000000}
\pgfsetstrokecolor{dialinecolor}
\draw (16.550407\du,17.875000\du)--(19.483363\du,17.875000\du)--(19.483363\du,12.941667\du)--(22.416319\du,12.941667\du);
}}
\pgfsetlinewidth{0.100000\du}
\pgfsetdash{}{0pt}
\pgfsetdash{}{0pt}
\pgfsetmiterjoin
\pgfsetbuttcap
{
\definecolor{dialinecolor}{rgb}{0.000000, 0.000000, 0.000000}
\pgfsetfillcolor{dialinecolor}
% was here!!!
\pgfsetarrowsend{latex}
{\pgfsetcornersarced{\pgfpoint{0.000000\du}{0.000000\du}}\definecolor{dialinecolor}{rgb}{0.000000, 0.000000, 0.000000}
\pgfsetstrokecolor{dialinecolor}
\draw (16.550407\du,17.875000\du)--(19.488363\du,17.875000\du)--(19.488363\du,22.800000\du)--(22.426319\du,22.800000\du);
}}
\pgfsetlinewidth{0.100000\du}
\pgfsetdash{}{0pt}
\pgfsetdash{}{0pt}
\pgfsetmiterjoin
\pgfsetbuttcap
{
\definecolor{dialinecolor}{rgb}{0.000000, 0.000000, 0.000000}
\pgfsetfillcolor{dialinecolor}
% was here!!!
\pgfsetarrowsend{latex}
{\pgfsetcornersarced{\pgfpoint{0.000000\du}{0.000000\du}}\definecolor{dialinecolor}{rgb}{0.000000, 0.000000, 0.000000}
\pgfsetstrokecolor{dialinecolor}
\draw (16.550407\du,17.875000\du)--(19.527530\du,17.875000\du)--(19.527530\du,17.891667\du)--(22.504652\du,17.891667\du);
}}
% setfont left to latex
\definecolor{dialinecolor}{rgb}{0.000000, 0.000000, 0.000000}
\pgfsetstrokecolor{dialinecolor}
\node[anchor=west] at (23.700000\du,14.625000\du){};
\definecolor{dialinecolor}{rgb}{1.000000, 1.000000, 1.000000}
\pgfsetfillcolor{dialinecolor}
\fill (34.250000\du,8.200000\du)--(34.250000\du,20.150000\du)--(40.500000\du,20.150000\du)--(40.500000\du,8.200000\du)--cycle;
\pgfsetlinewidth{0.100000\du}
\pgfsetdash{}{0pt}
\pgfsetdash{}{0pt}
\pgfsetmiterjoin
\definecolor{dialinecolor}{rgb}{0.000000, 0.000000, 0.000000}
\pgfsetstrokecolor{dialinecolor}
\draw (34.250000\du,8.200000\du)--(34.250000\du,20.150000\du)--(40.500000\du,20.150000\du)--(40.500000\du,8.200000\du)--cycle;
% setfont left to latex
\definecolor{dialinecolor}{rgb}{0.000000, 0.000000, 0.000000}
\pgfsetstrokecolor{dialinecolor}
\node at (37.375000\du,14.370000\du){Logger};
\pgfsetlinewidth{0.100000\du}
\pgfsetdash{}{0pt}
\pgfsetdash{}{0pt}
\pgfsetmiterjoin
\pgfsetbuttcap
{
\definecolor{dialinecolor}{rgb}{0.000000, 0.000000, 0.000000}
\pgfsetfillcolor{dialinecolor}
% was here!!!
\pgfsetarrowsend{latex}
{\pgfsetcornersarced{\pgfpoint{0.000000\du}{0.000000\du}}\definecolor{dialinecolor}{rgb}{0.000000, 0.000000, 0.000000}
\pgfsetstrokecolor{dialinecolor}
\draw (17.099456\du,5.925000\du)--(31.550000\du,5.925000\du)--(31.550000\du,14.175000\du)--(34.200119\du,14.175000\du);
}}
\pgfsetlinewidth{0.100000\du}
\pgfsetdash{}{0pt}
\pgfsetdash{}{0pt}
\pgfsetmiterjoin
\pgfsetbuttcap
{
\definecolor{dialinecolor}{rgb}{0.000000, 0.000000, 0.000000}
\pgfsetfillcolor{dialinecolor}
% was here!!!
\pgfsetarrowsend{latex}
{\pgfsetcornersarced{\pgfpoint{0.000000\du}{0.000000\du}}\definecolor{dialinecolor}{rgb}{0.000000, 0.000000, 0.000000}
\pgfsetstrokecolor{dialinecolor}
\draw (28.115336\du,12.941667\du)--(31.550000\du,12.941667\du)--(31.550000\du,14.175000\du)--(34.200119\du,14.175000\du);
}}
\pgfsetlinewidth{0.100000\du}
\pgfsetdash{}{0pt}
\pgfsetdash{}{0pt}
\pgfsetmiterjoin
\pgfsetbuttcap
{
\definecolor{dialinecolor}{rgb}{0.000000, 0.000000, 0.000000}
\pgfsetfillcolor{dialinecolor}
% was here!!!
\pgfsetarrowsend{latex}
{\pgfsetcornersarced{\pgfpoint{0.000000\du}{0.000000\du}}\definecolor{dialinecolor}{rgb}{0.000000, 0.000000, 0.000000}
\pgfsetstrokecolor{dialinecolor}
\draw (28.205214\du,17.891667\du)--(31.550000\du,17.891667\du)--(31.550000\du,14.175000\du)--(34.200119\du,14.175000\du);
}}
\pgfsetlinewidth{0.100000\du}
\pgfsetdash{}{0pt}
\pgfsetdash{}{0pt}
\pgfsetmiterjoin
\pgfsetbuttcap
{
\definecolor{dialinecolor}{rgb}{0.000000, 0.000000, 0.000000}
\pgfsetfillcolor{dialinecolor}
% was here!!!
\pgfsetarrowsend{latex}
{\pgfsetcornersarced{\pgfpoint{0.000000\du}{0.000000\du}}\definecolor{dialinecolor}{rgb}{0.000000, 0.000000, 0.000000}
\pgfsetstrokecolor{dialinecolor}
\draw (28.126163\du,22.800000\du)--(31.550000\du,22.800000\du)--(31.550000\du,14.175000\du)--(34.200119\du,14.175000\du);
}}
% setfont left to latex
\definecolor{dialinecolor}{rgb}{0.000000, 0.000000, 0.000000}
\pgfsetstrokecolor{dialinecolor}
\node[anchor=west] at (23.700000\du,14.625000\du){};
% setfont left to latex
\definecolor{dialinecolor}{rgb}{0.000000, 0.000000, 0.000000}
\pgfsetstrokecolor{dialinecolor}
\node[anchor=west] at (20.150000\du,4.933333\du){Alerts aggregated and formatted for Splunk};
% setfont left to latex
\definecolor{dialinecolor}{rgb}{0.000000, 0.000000, 0.000000}
\pgfsetstrokecolor{dialinecolor}
\node[anchor=west] at (12.716667\du,10.700000\du){Honeypot traffic forwarded};
% setfont left to latex
\definecolor{dialinecolor}{rgb}{0.000000, 0.000000, 0.000000}
\pgfsetstrokecolor{dialinecolor}
\node[anchor=west] at (12.716667\du,11.500000\du){    to localhost services};
% setfont left to latex
\definecolor{dialinecolor}{rgb}{0.000000, 0.000000, 0.000000}
\pgfsetstrokecolor{dialinecolor}
\node[anchor=west] at (10.858333\du,16.637500\du){};
% setfont left to latex
\definecolor{dialinecolor}{rgb}{0.000000, 0.000000, 0.000000}
\pgfsetstrokecolor{dialinecolor}
\node[anchor=west] at (10.858333\du,17.437500\du){  Public};
% setfont left to latex
\definecolor{dialinecolor}{rgb}{0.000000, 0.000000, 0.000000}
\pgfsetstrokecolor{dialinecolor}
\node[anchor=west] at (10.858333\du,18.237500\du){Interface};
% setfont left to latex
\definecolor{dialinecolor}{rgb}{0.000000, 0.000000, 0.000000}
\pgfsetstrokecolor{dialinecolor}
\node[anchor=west] at (10.858333\du,19.037500\du){};
\pgfsetlinewidth{0.100000\du}
\pgfsetdash{}{0pt}
\pgfsetdash{}{0pt}
\pgfsetbuttcap
{
\definecolor{dialinecolor}{rgb}{0.000000, 0.000000, 0.000000}
\pgfsetfillcolor{dialinecolor}
% was here!!!
\pgfsetarrowsend{latex}
\definecolor{dialinecolor}{rgb}{0.000000, 0.000000, 0.000000}
\pgfsetstrokecolor{dialinecolor}
\draw (2.991667\du,17.837500\du)--(9.874697\du,17.862754\du);
}
\definecolor{dialinecolor}{rgb}{1.000000, 1.000000, 1.000000}
\pgfsetfillcolor{dialinecolor}
\fill (4.125000\du,17.909167\du)--(4.125000\du,21.054167\du)--(7.077500\du,21.054167\du)--(7.077500\du,17.909167\du)--cycle;
% setfont left to latex
\definecolor{dialinecolor}{rgb}{0.000000, 0.000000, 0.000000}
\pgfsetstrokecolor{dialinecolor}
\node[anchor=west] at (4.125000\du,18.504167\du){};
% setfont left to latex
\definecolor{dialinecolor}{rgb}{0.000000, 0.000000, 0.000000}
\pgfsetstrokecolor{dialinecolor}
\node[anchor=west] at (4.125000\du,19.304167\du){Incoming};
% setfont left to latex
\definecolor{dialinecolor}{rgb}{0.000000, 0.000000, 0.000000}
\pgfsetstrokecolor{dialinecolor}
\node[anchor=west] at (4.125000\du,20.104167\du){  Traffic};
% setfont left to latex
\definecolor{dialinecolor}{rgb}{0.000000, 0.000000, 0.000000}
\pgfsetstrokecolor{dialinecolor}
\node[anchor=west] at (4.125000\du,20.904167\du){};
\pgfsetlinewidth{0.100000\du}
\pgfsetdash{}{0pt}
\pgfsetdash{}{0pt}
\pgfsetmiterjoin
\pgfsetbuttcap
{
\definecolor{dialinecolor}{rgb}{0.000000, 0.000000, 0.000000}
\pgfsetfillcolor{dialinecolor}
% was here!!!
\pgfsetarrowsend{latex}
{\pgfsetcornersarced{\pgfpoint{0.000000\du}{0.000000\du}}\definecolor{dialinecolor}{rgb}{0.000000, 0.000000, 0.000000}
\pgfsetstrokecolor{dialinecolor}
\draw (6.433182\du,17.850127\du)--(6.433182\du,9.770833\du)--(13.575000\du,9.770833\du)--(13.575000\du,7.700036\du);
}}
% setfont left to latex
\definecolor{dialinecolor}{rgb}{0.000000, 0.000000, 0.000000}
\pgfsetstrokecolor{dialinecolor}
\node[anchor=west] at (1.258333\du,0.637500\du){};
% setfont left to latex
\definecolor{dialinecolor}{rgb}{0.000000, 0.000000, 0.000000}
\pgfsetstrokecolor{dialinecolor}
\node[anchor=west] at (1.258333\du,1.437500\du){Network};
% setfont left to latex
\definecolor{dialinecolor}{rgb}{0.000000, 0.000000, 0.000000}
\pgfsetstrokecolor{dialinecolor}
\node[anchor=west] at (1.258333\du,2.237500\du){ Traffic};
\pgfsetlinewidth{0.100000\du}
\pgfsetdash{}{0pt}
\pgfsetdash{}{0pt}
\pgfsetbuttcap
{
\definecolor{dialinecolor}{rgb}{0.000000, 0.000000, 0.000000}
\pgfsetfillcolor{dialinecolor}
% was here!!!
\pgfsetarrowsend{latex}
\definecolor{dialinecolor}{rgb}{0.000000, 0.000000, 0.000000}
\pgfsetstrokecolor{dialinecolor}
\draw (40.548572\du,14.173477\du)--(46.058333\du,14.170833\du);
}
\definecolor{dialinecolor}{rgb}{1.000000, 1.000000, 1.000000}
\pgfsetfillcolor{dialinecolor}
\fill (40.658333\du,11.575833\du)--(40.658333\du,13.920833\du)--(43.708333\du,13.920833\du)--(43.708333\du,11.575833\du)--cycle;
% setfont left to latex
\definecolor{dialinecolor}{rgb}{0.000000, 0.000000, 0.000000}
\pgfsetstrokecolor{dialinecolor}
\node[anchor=west] at (40.658333\du,12.170833\du){};
% setfont left to latex
\definecolor{dialinecolor}{rgb}{0.000000, 0.000000, 0.000000}
\pgfsetstrokecolor{dialinecolor}
\node[anchor=west] at (40.658333\du,12.970833\du){To Splunk};
% setfont left to latex
\definecolor{dialinecolor}{rgb}{0.000000, 0.000000, 0.000000}
\pgfsetstrokecolor{dialinecolor}
\node[anchor=west] at (40.658333\du,13.770833\du){};
\end{tikzpicture}

}
\end{figure}
