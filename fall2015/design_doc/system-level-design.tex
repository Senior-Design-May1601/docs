\chapter{System Level Design}

\blindtext
\section{System Requirements}
When requirements are gathered, it is important to break them up into “functional” requirements, and “non-functional” requirements. It is necessary to note the difference between the two, as they are key to understanding what to develop, and how to accomplish a project. Functional requirements are often thought of as a description of a system’s behavior, and non-functional requirements elaborate performance characteristics of the system. In this section, a list of functional requirements, as well as non-functional requirements can be found.

\subsection{Functional Requirements:}
\begin{itemize}
\item The system shall interface with SSH, HTTP, and HTTPS protocols.
\item The system shall use a fake login page with key logger to record login attempts.
\item The system shall record logs of all connection attempts.
\item The system shall send alerts to administrative personnel upon detection irregularities that include, but are not limited to, increased ICMP traffic, port scans, and repeated attempts to connect to URLs.
\item A rule set must be included that allows for fine-tuning to accommodate new or changed rules.
\item The system shall contain a small intrusion detection system (IDS).
\end{itemize}

\subsection{Non-Functional Requirements:}
\begin{itemize}
\item The system must be low maintenance.
\item The system must be a standalone device.
\item The system must be low power.
\item The system shall allow the support of no fewer than 28 devices.
\item The system shall have an uptime of no less than 99%.
\end{itemize}

\section{Functional Decomposition}

The SCADA honeypot to be built for twenty-eight of Alliant Energy's powerplant substation will have seven main components that allow it to mimic and log potential attacks.  These components consist of a public firewall, an SSH server, a local webserver, an Ansible configuration manager, a log generator, a localized intrusion detection system, and a centralized Splunk server.  To gain a more complete understanding of how data is processed through each honeypot the next few paragraphs will describe the purpose and functionality of each component.

\subsection{Parts Breakdown:}
\begin{itemize}
\item Public Firewall
\newline
The public firewall also known as the public interface is one of two means of ingress to the system.  The firewall setup uses Linux IP tables to control the flow of traffic to only information coming in on ports 22(SSH), 80(HTTP), 443(HTTPS) and port 2222.  These ports are opened up for the reason that they are the bare minimum required to run the services of the onboard servers.  This give the illusion that the honeypot is a secure service and helps to obsucre its true intentions to a malicious user.  Note: The public firewall does not intereact with the second network interface card because it is monitored directly by the BRO Intrusion Detection System.

\item SSH Server
\newline
The SSH server setup after the public facing firewall is a mock server that only simulates the authentication process of the TCP/IP handshake where both parties exchange SYN and ACK packets to verify the user.  However the server setup on the honeypot is designed to terminate the handshake one the attempting user submits their credentials.  The SSH server then sends the username, password and other important information to the logger to be processed into one cohesive log.

\item Local Webserver
\newline
The local webserver implemented in the design of the honeypot system works in a similar fashion to the SSH server. Esentially if a malicious user attempts to connect via HTTP/HTTPS they will be directed to the webserver.  This false webserver acts as an inforamtion gathering tool for Alliants admin.  Once inside the firewall the user will be presented with a login page that asks for a username and password.  Once entered the information is logged and sent to the logger just like the SSH server.  Upon entering their credential the system will wait for a period of time before displaying that the credentials entered are not accepted.  This provides the ability to log mulitiple attempts to access the system through HTTP/HTTPS.

\item Ansible Configuration Manager
\newline
The Ansible Configuration Mangager is responsible for the setting up the initial configuration of each type of server and also pushing any patches that are necessary for the device to run properly.  Becuase of this it will only be accessible by network users with administrative rights.

\item Log Generator
\newline
The log generating component of the honeypot takes information from the all other components and formats them into an easy to read informative text file that can be sent to a central sever in Alliant Energy's data center.  These logs will cosistenly record normal network activity as well as any attacks that occur on the device.  The log format will be structured in accordance with Alliant Energy's choosing, but will contain information on the time the attack was inintiated, what type of attack it was, the user name and password, and other credentials.  This data will then be relayed back to a Splunk server at Alliant's main data center for analysis.

\item Splunk Server
\newline
The splunk server is a compontent in Alliant's data center that receives the logs generated in the honeypot and analyzes them to determine if an attack has happened according to Alliant's definition of an attack.  When an attack does occur the splunk server will contact an administrator via email to alert that an attack.  From this information the administrators can make  the appropriot respose to keep the network secure.

\item Bro Intrusion Detetction System
\newline
The Bro IDS software listens on a seprate network interface card than the firewall and the other servers within the honeypot.  This second NIC is unregulated and connects directly to the IDS, meaning that is allows all traffic in.  By doing this the hoenypot will be able to capture any other attack/malicious traffic that occur within the SCADA system.  However none of the other systems will be compromised becuase of this becuse each NIC is physically separate from each other and only connect at the logger which works the same way for either system.  

\section{System Analysis}
This project is routinely implemented in industry with high success rates. Several software implementations of SCADA honeypots already exist in the open source community. However, our implementation will be custom because the client has a secondary goal of including an IDS on the system. Creating the honeypot software from scratch will minimize the resources required from the platform. This is important because the computer needs to be a small standalone device. A Raspberry PI only has 512MB of RAM for example and an IDS typically uses a lot of memory and process cycles. Most risk can be mitigated with proper contingency planning. Installing an IDS on a small platform such as a Raspberry PI may not be feasible. If this is the case, developing a custom platform with greater processing power and more memory will be explored. If this isn't feasible, then the secondary goal will have to be canceled. However, the overall honeypot implementation carries a low risk of failure.
\newline
\newline
Alliant Energy requires service in twenty eight different locations. Each location will house a standalone minimal computer. At this stage of development, the Raspberry PI is the assumed device. This device is subject to change throughout the project development depending on implementation and client needs. However, as of the present date, the Raspberry PI is the best known device. An updated version of the plan may be submitted if a better hardware implementation is found. This cost analysis reflects an initial estimate. It does not reflect future maintenance costs.

\section{Block Diagrams}

\Blindtext
